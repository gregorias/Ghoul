\chapter{Motivation}
Topology creation and maintainance protocols are one of the first choices a
designer of a Peer-to-Peer system has to face. Distributed hash tables, such as
Chord or Kademlia, are frequently chosen due to their good performance.
Unfortunately those core protocols rely on correct behaviour of their
participants and are prone to malicious behaviour. A completely unprotected
system may be controlled or shutdown by even a modest adversary.

There exists a rich set of proposals which aim to secure DHTs, and yet most of
the popular and widely used Peer-to-Peer applications using DHTs, such as:
Bitorrent Kad, OpenDHT are not using them and are known to be insecure. I feel
that this is, because those solutions often suffer from faults, such as:
suboptimal performance, too stringent assumptions, i.e. reliance on central
authority and lack of good implementation.

This work aims to provide a DHT system that provides a secure DHT implementation
that protects the DHT against wide array of attacks, doesn't need to rely on
central, trusted authority and can self-recover in case an attack happens. This
implementation will done using Java, a popular language thanks to its
performance and portability, using tested software engineering practices that
provide elasticity and customazibility necessary to allow easy inclusion of the
library into an application.

[Currently used non-secure solutions: Yaca, Bittorrent Kad, Coral
ContentDistribution]

\documentclass{pracamgr}

%\usepackage[left=1in,top=1in,right=1in,bottom=1in,nohead]{geometry}

%\setlength{\parindent}{0pt}
%\setlength{\parskip}{1ex plus 0.5ex minus 0.2ex}

\usepackage[utf8]{inputenc}

%Fix for paragraph ended before \@tempa was complete
\let\zz\[\let\zzz\]

\usepackage{xltxtra}

%Fix for paragraph ended before \@tempa was complete
\let\[\zz\let\]\zzz

%%% fix for \lll
\let\babellll\lll
\let\lll\relax

%
\usepackage{polyglossia}

%newtheorem*
\usepackage{amsthm}

%mathbb
\usepackage{amsfonts}

\usepackage{amsmath} %split

\newcommand{\N}{\mathbb{N}}
\newtheorem*{defin*}{Definition}
\newtheorem*{theorem*}{Theorem}
\newtheorem{theorem}{Theorem}

\usepackage{msc}
\usepackage{algorithm}
\usepackage{algorithmic}
\usepackage{marginnote}

\usepackage{hyperref}
\usepackage{svg}
\usepackage{graphicx}

%For good-looking Java code
\usepackage{listings}
\usepackage{color}

\definecolor{dkgreen}{rgb}{0,0.6,0}
\definecolor{gray}{rgb}{0.5,0.5,0.5}
\definecolor{mauve}{rgb}{0.58,0,0.82}

\lstset{frame=tb,
  language=Java,
  aboveskip=3mm,
  belowskip=3mm,
  showstringspaces=false,
  columns=flexible,
  basicstyle={\small\ttfamily},
  numbers=none,
  numberstyle=\tiny\color{gray},
  keywordstyle=\color{blue},
  commentstyle=\color{dkgreen},
  stringstyle=\color{mauve},
  breaklines=true,
  breakatwhitespace=true,
  tabsize=3
}

\makeindex

\title{Ghoul: A security extension of the Kademlia DHT protocol}
\author{Grzegorz Milka}
\nralbumu{306320}
\kierunek{Computer Science}
\opiekun{Krzysztof Rządca, PhD\\
  Faculty of Mathematics, Informatics, and Mechanics\\
University of Warsaw}
\dziedzina{ 
11.3 Computer Science\\ 
}
\funding{
  Foundation for Polish Science, HOMING PLUS/2010-2/13 
}
\keywords{Kademlia, distributed system, SybilControl, robust random key
generation, Ghoul, dfuntest}
\klasyfikacja{
C. Computer Systems Organisation\\
C.2 Computer-communication networks\\
C.2.4 Distributed Systems\\
Distributed applications}
\polishtitle{Ghoul: Rozszerzenia bezpieczeństwa do Kademlii, protokołu rozproszonej
tablicy haszującej}

\begin{document}

\maketitle

\begin{abstract}
Topology creation and maintenance protocols are one of the first choices a
designer of a Peer-to-Peer system has to face. Distributed hash tables, such as
Chord or Kademlia, are frequently chosen due to their good performance.
Unfortunately these core protocols rely on correct behavior of their
participants and are prone to malicious behavior. A completely unprotected
system may be controlled or shutdown by even a modest adversary.

There exists a rich set of proposals which aim to secure DHTs, and yet most of
the popular and widely used Peer-to-Peer applications using DHTs, such as:
Bittorrent Mainline DHT, Kad, OpenDHT, are not using them and are known to be
insecure. We feel that this is the case, because these solutions often are found
to be impractical by developers.
Impracticalities include things such as: suboptimal performance, too stringent assumptions, and lack of good implementation.

This goal of this work is to design and implement an extensible DHT system
that is protected against wide array of attacks.
This implementation is done using Java using tested software engineering practices that provide elasticity and customizability necessary to allow easy inclusion of the library into an application.
\end{abstract}


\tableofcontents

\chapter{Introduction}

BitTorrent --- a popular file sharing protocol --- is used by millions of users
around the world, yet current implementations of its Distributed Hash Table
(DHT) protocol, which may be used to download file metadata, are insecure.
A modest attacker may abuse the Distributed Hash Table protocol to slow down the
metadata search, block access to some files, or even effectively shut down the
service for significant portion of its users.
Luckily for BitTorrent, DHT is only an optional mechanism for metadata
retrieval, but not every application may treat DHT as an optional mechanism.
DHT is a general and efficient tool that serves as a basis for Peer-to-Peer
applications.
A basis, which insecurity may break the entire application and which security is
hard to guarantee.

Peer-to-Peer (P2P) applications are applications which run on multiple computers in
such a way that each instance is treated equally.
P2P applications are in direct opposition to client-server applications, such as
used in, for example, web-browsing.
In client-server model client instances request data from the server, which
stores and serves the data.
Peer-to-Peer model is popular among applications that need to scale dynamically
with the size of their user-base without using application provider's resources
and applications for which users want privacy and independence from central
authority.
A good example of both of those factors at play is visible in BitTorrent.
A BitTorrent application, which wants to download a file, acquires a list of
other computers that are interested in that file (either want to download it or
have it).
Next, the application downloads missing parts of the file from hosts that have
those parts, while at the same time distributes parts of the file it has to
other hosts.
This kind of sharing approach means that with the increase of an interest in a
file the total bandwidth of hosts having the file also increases.
Since there is no central authority, no one may easily collect data about user
behavior (users do not have to authenticate themselves and may use anonymizing
tools, such as Tor) or shutdown the service.

A fundamental part of a Peer-to-Peer application is how different instances
communicate with each other.
Unlike the client-server model, in which there is a global known, always-present
authority to communicate to, Peer-to-Peer applications have to handle users
dynamically joining and leaving the network (this phenomenon is called churn)
and find a way to efficiently send messages between two arbitrary nodes.
This problem may be compared the problem of sending a message to an unknown
town.
We are in a town and we want to find out what other towns are in the country and
how to send a message to those towns.
However, we only know how to reach neighboring towns and we do not know the
entire map of the country.
Our neighbors are in the same situation so we have to establish a cooperation
protocol to help each other achieve our goals.

One of the roles of a Peer-to-Peer application is to define this cooperation
protocol for establishment of a neighborhood relation between two nodes
(equivalent of a road in our analogy; the entire map is called a topology) and
establishment of routes between two non-neighboring nodes.
A popular approach is to use structured topologies created by DHT.
A structural topology means that there are strict rules which nodes may be
neighbors and how messages should propagate.
Topology may also be unstructured, meaning non-strict (or even lack of) rules.
The benefit of the structural approach is that the structure that they create
has certain properties that guarantee that routes between non-neighboring nodes
will be relatively short and easy to find.

Unfortunately, the structural nature of the topologies created by DHTs and
their Peer-to-Peer character make them vulnerable to attacks.
Since topology is the basis of a Peer-to-Peer application, a successful attack
on the topology is often enough to shut down the application entirely.
One type of attack is a Sybil attack.
An adversary advertises itself as multiple, independent hosts (towns).
It makes us think that the network (the country) is populous, but in reality
every message is sent to only one host.
The adversary's host then has larger control over the network than it should
have and may, for example, eavesdrop, poison (maliciously change its content), or
drop messages.
Another is an Eclipse attack.
An honest host is surrounded by malicious hosts; whenever anyone wants to find
the honest host then their queries are blocked by the adversary.

Peer-to-Peer applications using DHTs often leave the DHT woefully unprotected.
We'll show that, for example, BitTorent's DHT --- MainlineDHT --- is especially
vulnerable to even a modest adversary.
We think there are two main reasons for this lack of care.
First, DHT security mechanisms are seen as unpractical: either they require
unreasonable resources (such as access to a social network), or have significant
performance overhead.
Second, most security mechanisms are not backward compatible with existing
implementations.

In this thesis we design Ghoul --- a prototype secure DHT ---
as an answer to the impracticality concerns.
We show that it is possible to have a reasonable DHT protocol, that is secure
against common types of attacks, does not require additional resources from the
user (such as access to user's social network), and whose message complexity of
common operations is at worst logarithmic with regard to the size of the
network.
We also argue that some backward compatible security extensions, such as those
used in Kad DHT (a DHT used by eMule file-sharing application), are not
bullet-proof and with incoming wide adoption of IPv6, they might be rendered
void.

Additionally, to ensure high quality of the prototype, we designed and
implemented dfuntest --- a framework for distributed testing of Peer-to-Peer
applications.
Dfuntest was originally developed by the author to test Kademlia application in
Planet-lab environment, but its design later proved to be flexible enough to be
a general framework for any P2P application.

We hope that Ghoul will prove to be an useful tool to application developers in
creation of secure Peer-To-Peer applications.

\section{Outline}
In chapter \ref{ch:threats} we present the problem domain.
We define what is a Distributed Hash Table (DHT) and evaluate its typical
weaknesses.
We discuss the main types of security threats that DHTs are vulnerable to.
Finally, we review security mechanisms proposed in related work.
In Chapter \ref{ch:description} we describe the protocols used in Ghoul and
discuss Ghoul's strengths and weaknesses.
In Chapter \ref{ch:implementation} we explain how Ghoul is built and run as a
stand-alone application.
We focus on presenting tools and techniques used to guarantee high-quality code.
In Chapter \ref{ch:dfuntest} we present dfuntest --- a framework for testing
distributed peer-to-peer applications.
In Chapter \ref{ch:evaluation} we show how we test correctness and
performance of Ghoul.
In Chapter \ref{ch:conclusion} we summarize our contribution and state future
plans towards the prototype we have developed.

\section{Contributions}
This thesis has the following contributions:
\begin{itemize}
  \item Design of Ghoul --- a security extension of the Kademlia protocol that
    protects against malicious attacks.
  \item Ghoul's prototype --- Java implementation of the Kademlia protocol with
    message extensions, certificate management, and registration mechanisms.
  \item Design and implementation of dfuntest --- a framework for automation of
    distributed tests of Peer-to-Peer applications.
\end{itemize}

\chapter{DHT, its weaknesses, and defense mechanisms}
\section{DHT overview}

  Distributed hash table is a type of peer-to-peer system providing topology
  creation and maintainance protocols. They are typically designed for and
  tightly coupled with data storage protocols for key-value pairs, hence the
  hash table part in the name. Popular DHT designs include: Chord, Pastry,
  Kademlia, CAN.

  A structure of a typical DHT consist of a few main components. First is a
  keyspace, a collection of identifiers such as 160-bit bitstrings. Those keys
  are an abstract identification of a node and also serve to partition ownership
  in the network. Next is an overlay network that is a topology induced by
  routing tables, neighbourhood tables kept and maintained by each node
  containing pairing between an abstract key and a network address. Each DHT
  defines protocols for table maintainance and search of nodes that are not
  neighbours.

  This kind of structure has certain desired properties. First of is that DHT is
  completely decentralized, every peer is equal and protocol do not require any
  for of central coordinator. DHTs are also scalable, in most protocols the size
  of the routing table is on the order of $O(\log n)$, where $n$ is size of the
  network and similarly for message count complexity in node search protocol.
  Third of all DHTs are fault tolerant, in fact their protocols are for the fact
  that there might be significant churn in the network, churn is the rate node's
  join/leave speed, and the routing table should be updated often enough to
  counteract that.
 
  \subsection{Kademlia}
  Kademlia \cite{may02} is popular choice among many peer-to-peer applications,
  especially used in bittorrent and edonkey file-sharing applications.

  A node in kademlia network is represented by an n-bit binary key (most popular
  choices are $n = 128$ or $n = 160$). Kademlia defines a distance in the
  keyspace as the xor function. That is distance between two keys $x, y$ is

  \[ d(x, y) = x \oplus y\]

  The routing table is a collection of $n$ buckets, where each bucket has
  maximum size $k$, a system-wide constant. 
  $i$'th bucket holds active and recently seen nodes such that the first most
  significant bit that is different from routing table's owner is at position
  $i$. More formally the $i$'th bucket of node's with key $x$ holds nodes with
  keys $y$ such that $2^i \leq d(x, y) < 2^{i+1}$.

  Kademlia's node should implement 4 RPC queries:

  \begin{description}
    \item{\verb|FIND_NODE|} This query is parametrized by the searched key. The
      response should consists of up to $k$ keys closest the searched key in
      queried node's routing table or itself.
    \item{\verb|PING|} This query serves as a heart-beat and node's liveness
      check. It should be answered with a \verb|PONG| message.
    \item{\verb|PUT_KEY|} Orders the recipient to store given (key, value) pair
      in its local storage. Usually the sender first searches for nodes close
      to the pair's key.
    \item{\verb|GET_KEY|} Just like \verb|FIND_NODE| this query contains
      searched key. The recipient should return $k$ closest neighbours plus a
      (key, value) pair if it's present in its storage.
  \end{description}

  The main protocol in kademlia is the iterative node search protocol
  \ref{fig:node_search_alg}.

  \begin{figure}
    \label{fig:node_search_alg}
    \begin{algorithmic}[1]
    \STATE Take $k$ closest nodes to key $y$ in the routing table.
    \WHILE {There are unqueried nodes in top $k$ closest nodes}
    \STATE Send \verb|FIND_NODE| queries to unqueried nodes in top $k$ 
    nodes closest to $y$, such that there are up to $\alpha$ parallel queries.
    \STATE $e \leftarrow$ response or timeout
    \IF{$e$ is a response}
      \STATE Change state of the node to queried. Add its response to the list
      of
      nodes under consideration.
    \ELSE
      \STATE Remove timed-out node from consideration
    \ENDIF
    \ENDWHILE
  \end{algorithmic}
    \caption{Node search protocol for key $y$}
  \end{figure}

  The protocol for finding stored data associated with a key is structurally the
  same. Only instead \verb|FIND_NODE| \verb|GET_KEY| is sent and reply contains
  stored data if any.

  Because the distance metric is symmetric kademlia does not require specialized
  routing table maintainance messages. Instead every with every message receipt
  a node adds the sender to its routing table if there is space. Otherwise it
  sends a \verb|PING| message to least recently seen node in given bucket, if it
  times-out than it is replaced by the new neighbour.

\subsection{Vulnerabilities}
  DHTs suffer from various vulnerabilities, which mostly stem
  from two diffuclties:
  \begin{enumerate}
    \item Node identity is a virtual concept and lacks connection to the real
      world. DHTs as a distributed storage space rely on data and message
      redundancy and cohesiveness of the topology. Both are easily broken if an
      attacker may either arbitrarily place itself in the topology (Eclipse
      attack) or get as many nodes as it wants (Sybil attack).

    \item Each node only maintains a local view over the network, which does not
      provide enough context to detect and prevent any malicious behaviour.
  \end{enumerate}

\section{Eclipse attack}
  Attacks on DHT can have 3 different goals:
  denial of service - the attacker aims to either stop the entire system from
  functioning or deny access from or to a part of it,
  subversion - the attacker wants to provide false data to other nodes or
  control them,
  information - the attacker wants to gather information about some nodes
  activity.

  Those goals are greatly facilitated if an attacker controls the routing tables
  of other nodes. An attack which poisons routing tables of other nodes by
  disproportionately filling them with entries to colluding malicious nodes is
  called an eclipse attack. The name comes from one of its effects, if a node's
  routing table is filled with malicious nodes it is effectively partitioned
  from the rest of the network.

  \subsection{Prevention mechanisms}

  Sit and Morris \cite{sit02} have been among the first to discuss general
  guidelines against preventing attacks in DHT networks. They advise desinging
  the system with verifiable constrainst so that a node can distinguish
  malicious behaviour from honest one and provide a proof to other nodes.
  Mechanisms, mainly data and node look up, should have a built-in redundancy so
  that a malicious node presence can be avoided in at least one path of
  execution. The lookup itself should be interactive and allow for clear view of
  progress, which allows a node to verify its correctness and initiate
  redundancy protocols if necessary. Finally, as in any distributed system,
  points of single responsibility should be avoided for obvious reasons.
  
  \subsubsection{Constrained routing table}
  Castro et al. \cite[p. 20]{urd11} have modified Pastry with additional
  constrained routing table and back up routing mechanism to protecct against
  eclipse attacks. If node key's assignment is truly random and routing table
  entries are narrowly specified then it is difficult for an attacker to inject
  malicious entries. DHTs such as Kademlia and Pastry are therefore more
  vulnerable than Chord.

  \subsubsection{Node scrambling}
  A different method to mitigate an eclipse attack is to constantly move node's
  ids in unpredictable ways so that it is infeasable to eclipse a local
  keyspace. Such solution is proposed by Condie et al. \cite[p. 21]{urd11}.
  Condie shows that a periodical random, impredictable node reassignment
  augmented by techniques which mitigate negative churn effects, that is
  grouping nodes into different timeout groups so that timeouts are not
  synchronized and precalculation of future routing table, significantly
  improves Castro's solution.

  Awerbach and Scheideler \cite{awe10}, while proposing a robust, distributed
  random number generator propose a scheme of global node rearrengment on each
  join node. In such a scheme once a join node they show that using a robust
  cuckoo rule to move $O(\log n)$ nodes makes the network resilient to
  join-leave attacks.

  \subsubsection{Topology graph analysis}
  Singh et al. \cite[p. 23]{urd11} observe that if malicious nodes form only a
  fraction of the network then during an Eclipse attack their in-degree must be
  higher than average in-degree of honest nodes. Therefore they propose a
  protocol which limits in-degree of nodes. Each node $x$ maintains an
  additional backpointer table, which contains nodes having $x$ in their routing
  tables. Every node $y$ periodically checks backpointer tables of their
  neighbours and if they are larger than a global constant or does not contain
  $y$ then it removes $x$ from its routing table.

  This scheme requires the backpointer check to be anonynomous, therefore the
  Scheme uses a common technique of onion routing using anonymizer network to
  achieve that. Anonymizer are not assumed to be honest and therefore it
  possible that a honest node will be labeled as malicious with small
  probability.

  This scheme is effective at limiting eclipse attacks, but it requires the
  global in-degree bound to be tight, especially for large networks, which may
  decrease the efficiency of base protocol.

\section{Sybil attack}
  DHTs routing mechanisms rely on the assumption that nodes are placed uniformly
  inside its key space and only a fraction of the network may fail, either due
  to churn or node failure. In case of eclipse attack or routing attack
  prevention mechanism they additionally assume that only a given portion of the
  network may be malicious. Those assumptions can be easily violated in an open
  DHT.

  Firstly, key generation is not verifiable. There is no way for a node to
  verify whether given key has been generated randomly. It is possible to use a
  statistical test to verify uniformity constraints, but they are often not
  enough to accuse specific node of malicious behaviour.

  Secondly, DHTs operate on an ephemeral sense of identity represented by a key.
  A Sybil attack happens when an attacker uses a single physical computer to
  represent multiple DHT nodes increasing his power without incurring additional
  cost. Since most protocols rely on honest nodes having a majority, a Sybil
  attack can shut down the system by even a modest attacker.

  Preventing a Sybil attack requires a gatekeeper or, more accurately, a
  bouncer. A protocol which prevents a single physical node to enter more than
  once and a way to throw out offenders. Achieving that in a fully distributed
  way is difficult. It is not possible to achieve adequate security if nodes
  only have access to their immediate surroundings. Such a limited view lacks
  necessary information to distinguish between bening and malicious behaviour.
  Additionally it's much easier to simply eclipse a node. Protocols that
  overcome this obstacle by sharing information globally do so at the cost of
  performance and scalibility. Such cost usually makes a DHT unpractical for
  larger networks. That's why most practical solutions make use of a central
  authority. This authority serves as a trusted entry point to the network, it
  generates signed random keys for newcomers and checks global constraints.

  Central authority has well-known drawbacks. Most of them stem from the fact
  that such authority represents a single point of failure. A failure of this
  authority may cause shut down of the network. Compromised authority is a
  security risk. It also acts negatively on scalability. Apart from
  technological disadvantages central authority is often an organizational
  problem. It forces an agent to maintain it in order to provide a service, this
  is done reluctantatly since it makes the agent responsible for maintaining the
  service and also Peer-to-Peer user community may be distrustful over
  centralized solution.

\subsection{Prevention mechanisms}

  The conclusion that only central authority solutions are practical for
  prevention of Sybil attacks was first deduced by \cite{dou02}, which also is
  the first paper studying this class of attacks \cite[p. 5]{urd11}

  [Tutaj jest praktycznie to co w \cite{urd11}]
  \subsubsection{Centralized gatekeeper}
  Castro et. al. \cite{cas02} agrees with \cite{dou02} conclusion and proposes
  using a set of central trusted authorities which certify random key and public
  key pair. They suggest adding an IP address to the certificate, so that the
  certificate would be binded to physical host. Unfortunately this solution is
  known to be problematic. First of all IP address can be spoofed and with IPv6
  privacy extensions changed al together. Second of all it would cause a problem
  for users behind NAT.

  Centralized gatekeeper gives much flexibility in implementing an
  authentication and sybil attack prevention scheme. For example there may be
  required a small monetary payment for entry into the DHT network or a
  computational puzzle may have to solved.

  \subsubsection{Distinction based on network footprint}

  Without centralized authority network peers would have to register new nodes
  in a distributed way. To distinguish distinct identities from same ones they
  may use network characterestics. Such characteristic are usually easy to check
  in the sense that any node may do it, giving low barrier of entry to the
  network, but it carries an assumption that they are unfalsifiable, which is
  not always the case.

  A simple ID calculation based on IP address and port number is proposed by
  Dinger and Hartensteing \cite[p. 6]{dou02}. They propose that a node would
  register $r$ Chord registration nodes whose position is calculated from hash
  of node's ID. If majority of registration nodes accept the new node then it is
  said to be accepted. A node registration fails if number nodes with given IP
  address exceeds a constant $a$. This solution requires all nodes to confirm
  validity of their neighbours.

  IP addresses may be spoofed and cause a problem for NAT users. Wang et. al.
  \cite[p. 7]{dou02} proposed adding router's IP and MAC address and RTT
  measurements between the node a designated set of landmarks. A similar idea
  may be found in Bazzi and Konjevod \cite[p. 7]{dou02} where nodes are put into
  $d$-dimensional space based on physical location of a node and a node are said
  to be distinct if their coordinates are far away enough.

  \subsubsection{Distinction based on social network}

  Another class of defenses uses a social network, a graph of trust
  relationships established by humans, to discover or mitigate a Sybil attack.
  In such defenses it is assumed that each node can verifiably identify itself
  as a node in the social network and that each edge forms a trust relationship
  which means that two people are friends and that they trust each other not to
  launch a Sybil attack. If any node was to launch a Sybil attack then the
  social network graph will exhibit special properties which may be detected. 

  Although such schemes show good performance, both in terms of application
  overhead and detectability, they are difficult to apply in real world
  applications.  [TODO aplikacja musi skądś ten graf wziąć co często jest
  niemile widziane lub niepraktyczne]

  SybilGuard \cite{hai06} mitigates effects of a Sybil attack by limiting the
  number of sybil nodes and recognized sybil groups. It defines the edges
  between honest nodes and sybil nodes as attack edges and notes that in a
  social graph there can be only few such edges. Then it allows a node to
  partition its neighbours into classes and guarantees with high probability
  that a number of classes that contain a Sybil node is limited by the number of
  attack edges. Although the scheme does not detect Sybil nodes it gives a
  guarantee that with a sufficient redundandancy an attacker will not be able to
  intercept a query or manipulate data.

  \subsubsection{Puzzles}
  SybilControl\cite{li12} is a distributed scheme for sybil prevention using
  computational puzzles. Every node is supposed to periodically generate a
  puzzle for itself based on puzzles of its neighbours and solve. Such puzzle
  is verifiable by any node using novel multi-hop verification and serves as a
  proof-of-work. Nodes that can not present an up-to-date proof-of-work are
  suspected of being a Sybil identity and removed from routing tables. This
  effectively bounds a number of sybil identities an attacker can create.

  Computation puzzles have the advantage of being flexible. They can be done by
  any node, provided it has enough resources and can also scale their
  difficulty. It is also much harder for an attacker to fake solving them or use
  social engineering like in social network solutions.

\section{Discussion}
  Eclipse and Sybil attacks do not usually constitute the goal of an attack, but
  are used to facilitate it. Such goal are usually to deny access to a resource
  or to fake it.

  Attacks that aim to fake possession of requested resource are best counter
  with cryptography. Using signatures, self-certyfying data, or Merkle hash
  trees, like in bittorrent, is an effective and efficient deterrent.
  To protect against malicious nodes dropping data or routing information the
  system needs to use replication and redundant routing. 

  Replication against malicious opponent is trickier than with her. There are
  two main strategies for replication: Place given resource at nodes with key
  close to the key of the resource or spread it over the network \cite[p.
  38]{urd11}. The first strategies allows for more efficient retrieval in DHTs
  such as Kademlia or Pastry, but is also more vulnerable to localized eclipse
  attack. Both strategies are easy to attack if an attacker has ability to
  choose her identifier arbitrarily and therefore a secure DHT requires secure
  node id assignment.

\section{Examples of real attacks}
These are not just theoretical divagations, but discussed attacks are really
used in practice and are effective.

\paragraph{Hijacking attack}
\cite{wan08} describes an attack on Kad network as implemented by eMule
file-sharing application. This is a version of a routing table poisoning attack
in which the attacker abuses lack of authentication and admission of control
which allows any node to advertise as any other. The is split into two phases:
\begin{enumerate}
  \item In the preparation phase the attacker sets up N uniformly distributed
    virtual nodes which gather routing table information of their neighbours.
  \item In the attack phase each attacking node send to its neighbours bogus
    updates of the form \verb|(ID_B, IP_A)\verb|, where \verb|ID_B| is the key
    of existing victim node present in the recipient's routing table and
    \verb|IP_A| is the attacker's IP address.
\end{enumerate}

Authors show that with just 100Mbps bandwith they were able to successfully
deny 80\% of keyword queries after 1-hour long preparation phase with a
network with 16000 nodes. Unfortunately, for the attacker, this cost is ongoing
since the attacking node needs to be responsive.

Authors also provide a version of this attack where instead of providing
attacker's IP the IP of the attacked node is used. Emule at the time did not
provide any check against this behaviour so it would be effective in blocking
any effective look up. They also calculate that in this version it would take
around 200Mbps of network speed to attack the entire 1 million eMule network
with the result of blocking up to 75\% queries.

More recent versions of eMule contain basic protections against attacks, such
as \cite{tim11}:
\begin{description}
  \item{flood protection} The number of messages received from one IP in a given
    time-frame is limited.
  \item{IP limitations} Only one node with given IP can be present in the
    routing table.
  \item{IP verification} Before a contact is added a three-way handshake is
    performed.
\end{description}

While those protections defend against small and unsophisticated attack it must
be noted that the Kad network is still vulnerable to moderately determined
attacker.

\paragraph{Feasibility of Sybil attack}
Although Sybil attack is more expensive than above mentioned Hijacking attack
\cite{tim11} has shown that in Mainline DHT implementation of Kademlia used in
Bittorrent it still easy. In their experiment they were able to fill neighbours
routing table with 160 Sybil nodes within 5 minutes after the attack phase has
begun. While similar attack would be impossible in newer versions of Kad the
only limitation is the fact that only one IP address is used.

\paragraph{Measurements of present attacks}
It is very easy to launch a Sybil attack in Mainline DHT and such an attacks has
distinctive footprint. Wang \cite{Wan12} has set up honeypots in the Mainline
DHT to detect and analyze possible Sybil attackers. He discovered an attacker
using a hijacking method described earlier running on Amazon's cloud who
maintains over 300000 thousand Sybil nodes. This attacker didn't seem to do much
and might have been just monitoring the network. A different attacker was
impersonating specific keyspace ranges and had distinctive IP address belonging
to international ISP service. He was redirecting queries from inside its domain
to other nodes inside it, localizing the traffic.

Authors conclude that although those attacks didn't seem to do much global harm,
the easiness with which they are executed proves that there are no costly
barriers to launching a successful attack to disable majority of DHT queries.

\chapter{Ghoul system description}
\section{Goals}
I decided to build Ghoul to fill a niche of secure open-source DHT network
implementation. In my experience it is difficult to find a good open-source
implementation of Kademlia. By good I mean one that satisfies all of the
following criterias:

\begin{description}
  \item{Simplicty} Implementation of Kademlia should do only one thing: operate
    according to the Kademlia protocol.
  \item{Extensibility} If I want to extend the network protocol with a NAT
    traversal mechanism I should not be required to modify large portion of the
    library network communication stack.
  \item{Documentation} An open-source project should have maintained
    documentation to help others contribute/use it.
  \item{Robustness} Implementation should not break after 1 hour of usage on
    small sized network in Planetlab.
\end{description}

I couldn't find an implementation that satisfies all of the above in Java and
none that would satisfiaibly claim to be secure in any language.

Solutions used in KAD network and designed for Mainline DHT use IP address to
prevent Sybil attack and arbirtary ID choice, this works against unsophisticated
attacker, but has serious problems stemming from the fact that IP was not
designed for this use case. IP spoofing, IPv6 privacy extensions and lack of
anonimity are serious concerns that either disarm such solution or introduce a
different serious flaw. I feel that lack of adoption of security measures
that are present in the research literature stems from the stereotype that such
measures are either hard to implement, slow, hard to satisfy their dependencies,
or needlessly complicate usage of the network. I wanted to find whether it was
possible to design and build DHT that satisfies following constraints:

\begin{description}
  \item{Security} The DHT should be resistant to common Sybil and eclipse
    attacks.
  \item{Performance} The DHT should be as performant as the original protocol
    where possible, especially in common use cases of finding a node and data
    storage. In other cases the time of operation should be practical, meaning
    that time or message complexity should not be greater than $O(\log n)$ for
    network of size $n$.
  \item{Simplicity} It should be simple to use. User should not be forced to
    perform an extensive set up, or configure a large file to just start the
    DHT.  The entire scheme should be simple to describe. User or developer
    should be able to understand how the protection schemes work and be able to
    reason how it would behave in his scheme. It should also be simple
  \item{Extensibility} The application develop should be able to extend the
    system with his own security mechanism easily if so desired.
\end{description}

When building the system I assume standard threats and network conditions that
is:

\begin{itemize}
  \item An attacker may introduce colluding nodes into the system, but has
    limited computing and communication power.
  \item For network communication there is point-to-point message passing
    primitive which may fail to deliver.
  \item Nodes may fail arbitrarily.
\end{itemize}

I believe such system is possible to accomplish and in this chapter I present
its design.

\section{Overview}
Ghoul system is built on top of Kademlia DHT protocol. Each node in the system
is an independent agent which acts as a Ghoul node and communicates with other
nodes using messages sent over network transmission.

There two kinds of nodes in the system: a DHT node and a registration authority.
The DHT node is the Kademlia node extended with security measures. It initaties
and maintain its routing table and anwers queries. Additionally each node is
responsible for maintaining the security of the entire network by participation
in P2P certificate revocation propagation, maintainance of its security
certificate, and answer in security queries.

A small set of nodes, called registration nodes, is responsible for faciliting
node’s join into network by using distributed random key generation and
certificate issuance. Registration nodes are distributed version of central
authority, with a twist that they a few of them might be malicious. Their
existance is motivated by the fact that it is very hard to guarantee random key
generation and verification in distributed setting, on the other hand I want to
provide administrative scalability where application's owner may allow other
parties to provide their registration node's for the protocol and not worry
whether they have pure intentions as long they do not collude.

\section{Data structures}

\subsection{Public key cryptography}
Ghoul uses public key cryptography for node authentication and certification.
Every dht node and registration node generates a public/private key pair on its
first entry. The private key is always kept private, but public can and should
be propagated. Additionally public keys of registration nodes are known globally
by all nodes.

To participate in the network each DHT requires a set of \textbf{certificate}.
Certificate is a tuple of: (\texttt{NODE\_PUB\_KEY}, \texttt{NODE\_DHT\_KEY},
\texttt{EXPIRATION\_DATETIME}) signed with a registration's node private key
such that it is known which node signed it. Such certificate is given by the
registration node together with \texttt{NODE\_DHT\_KEY} as a proof that that
node is valid and its key was generated randomly by the registration node or in
a disitributed way by a set of registration nodes. If the key is generated by a
group of registration nodes then all of them provide also the certificate.

We say that node is valid if it has at least $m$ non-expired, non-revocated
certifacates from different hosts. The parameter $m$ should be set to such a
value that we are sure that there might at most $m-1$ malicious registration
nodes.

Every node maintains a store for valid certificates that it has received. This
store may limited in size. In such a case a node may remove certificates last
seen node's or certificates close to expiration. Additionally every node in the
routing table must have a valid certificate in the store, otherwise it should be
removed. If a neighbours certificate is close to expiration than it should be
sent a \texttt{PING} to get a new certificate.

\subsection{Messages}
Ghoul extends each DHT message with following fields:
\begin{description}
  \item{\texttt{certificateRequest}} Signifies that a response to this message
    should contain a list of certificates. This is field is set when the
    recipient is considered to be a neighbour or potential neighbour and either
    its certificates are not present, not valid, or close to expiration.
  \item{\texttt{certificates}} A list of valid certificates provided by the
    registration authority.
  \item{\texttt{signature}} Signature signed by the sender's private key.
\end{description}

Just like normal Kademlia messages are used for routing table maintainance so
similarly in Ghoul we attach security information to them. Only the list of
certificates may have a larger size (each certificate has a size of $\approx$
1KB), but it is sent only sporadically.

Each one of those additional fields is optional. If the signature field is not
provided it means that the node is not interested in being part of the network
and instead just uses it to find information. That is considered normal
behaviour, such node is not added to the routing table (it may have possibly not
even provided its key information), and is simply responded with appropiate
response. Otherwise every node should always sign its messages.

The \texttt{certificateRequest} field is set if all of the following conditions
are met:

\begin{itemize}
  \item The recipient's certificate is not present in the store or is close to
    expiration.
  \item The recipient is in the routing table or may be considered a candidate.
\end{itemize}

If the \texttt{certificateRequest} field is set then the recipient should attach
list of certificates to its response or if no response would be normally given
then it should respond with a \text{PONG} message.

If a response to the message with the \texttt{certificateRequest} does not have
a valid list of certificates then it is discarded.

\section{Protocols}
\subsection{Challenge protocol}
In some neuralgical and vulnerable points in the protocol we may intent to limit
a number of requests a node can make using cryptographic challenges. This
achieves 2 goals: It significantly increases of a cost of an attack which relies
on number of queries, such as a Sybil attack, and provides a throttle mechanism
to prevent a node overload.

Whenever the protocol requires the node sending a query or a request to provide
a proof-of-work for this request the challenge protocol is used
(\ref{fig:chal_prot}).

\begin{figure}
\begin{msc}{Challenge protocol}
\setlength{\instdist}{9cm}
\setlength{\envinstdist}{3cm}
\declinst{client}{C}{Client}
\declinst{server}{S}{Server}
\mess{Send challenge request}{client}{server}
\nextlevel
\action*{
\begin{minipage}{4cm}\centering
Choose a random nonce\\
$r\in\{0,1\}^*$, difficulty $p \in \N$, $t := \text{current time}$.
Save the nonce into active puzzles list. If the answer does not arrive withing
time $T_0$, remove the nonce from the list.
\end{minipage}}{server}
\nextlevel[11]
\mess{Send challenge $m_2 := (t, r, p)_{s_{pr}}$}{server}{client}
\nextlevel[1]
\action*{
\begin{minipage}{4cm}\centering
  Find $s$ such that $H(r || s)$ has $p$ last bits equal to zero.
\end{minipage}}{client}
\nextlevel[4]
\mess{Send solution $m_3 := (m_2, s, c_{pub})_{c_{pr}}$}{client}{server}
\nextlevel
\action*{
\begin{minipage}{4cm}\centering
Check the validity of the solution. Remove the nonce from active puzzles list.
\end{minipage}}{server}
\nextlevel[4]
\end{msc}
  \caption{Challenge protocol}
  \label{fig:chal_prot}
\end{figure}

The validity check consists of: veryfying signature of $m_2$, checking whether
nonce is in active puzzles list, checking time, veryfying the solution.

$m_3$ serves as a proof-of-work and may be accompanied by the request to the
server.

Note that this protocol requires the server to use some space in order to save
the nonce. Depending on the size of a timeout and number of possible request per
second this may provide a denial-of-service vulnerability. If this is a possible
threat the application designer may choose to omit saving and nonce checking and
instead save the nonce after the puzzle has been saved for time $T_0$ and later
check whether it given nonce is not in the list. [TODO perhaps this is much
better solution and should be the default. It prevents repeat message attacks]

\subsection{Robust distributed random key generation}

In Ghoul honesty of registrars is not assumed. In case of a single centralized
registrar such a node would greatly facilitate in an eclipse attack by
assigning to colluding nodes keys close to a target. Therefore a distributed
scheme is needed.

Here  I will present a distributed protocol which gerates a random key that is
unbiased if at least one node is honest. More formally:

\begin{theorem}
  Given a list $R = \langle r_1, \ldots, r_n \rangle$ of $n \geq 2$ nodes known
  to all nodes in $R$ the random key generation protocol generates an unbiased
  random key known to all participants if at least one node is honest and all
  nodes respond according to the protocol in a timely manner. Even in the case
  of a byzantine failure then all nodes that generate a key generate the same
  key that is unbiased if at least one node is honest.

  The chance that the malicious participant may manipulate the protocol so that
  this theorem does not hold is cryptographically negligible.
\end{theorem}

The protocol uses bit-commitment. Bit-commitment is a cryptographic protocol
such that:

[TODO bit-commitmen description

The protocol that satisfies those properties is defined in \ref{fig:key_gen_alg}.

It consists of 3 phases:
\begin{enumerate}
  \item Commit and broadcast random number used to generate the key. This phase
    is used so that every node commits to its random number and can not change
    it later.
  \item Broadcast all received messages to others. This step ensures that every
    honest node has received the same set of messages and will generate the same
    key. If any node has sent inconsistent messages to honest nodes
    than this will be detected here.
  \item Broadcast key to unlock commitment.
\end{enumerate}

One may recognize that this protocol consist of first two phases of a 3-phase
byzantine failure protection protocol. Which is why it has its nonmalleability
properties.

\begin{figure}
  \begin{algorithmic}[1]
  \STATE $k_i \leftarrow $ random key
  \STATE $(h_i, s_i) \leftarrow $ bit commitment string and solution commiting
  $k_i$
  \FOR{$j := 1$ \TO $n$, $j \neq i$}
    \STATE Send $(h_i)_{i}$ to $r_j$
  \ENDFOR
  \STATE Wait $T$ seconds till all messages of the form $(h_j)_j$ are received
  from other nodes. If timeout happens then abort.
  \FOR{$j := 1$ \TO $n$, $j \neq i$}
    \STATE Send $\left( (h_1)_1, \ldots, (h_n)_{n}, k_i, s_i\right)_i$ to $r_j$
  \ENDFOR
  \STATE Wait $T$ seconds till all messages of the above form are received
  from other nodes. If timeout happens then abort. If any message is not
  coherent to the rest then abort.
  \FOR{$j := 1$ \TO $n$, $j \neq i$}
    \IF{$b_j$ is not a bit commitment of $k_j$}
      \STATE Abort protocol.
    \ENDIF
  \ENDFOR
  \STATE $k \leftarrow k_1 \oplus \ldots \oplus k_n$
\end{algorithmic}
  \caption{Distributed key generation algorithm generating random key $k$}
  \label{fig:key_gen_alg}
\end{figure}

An example of this protocol for 3 registrars is shown in
\ref{fig:key_gen_example}.

\begin{figure}
\begin{msc}{Distributed key-generation}
\setlength{\instdist}{5.5cm}
\setlength{\envinstdist}{3cm}
\declinst{ra}{}{RA}
\declinst{rb}{}{RB}
\declinst{rc}{}{RC}
\action*{
\begin{minipage}{4cm}\centering
  Generate random key $r_A$ and generate bit-commitment string from that key
  $h_A$.
\end{minipage}}{ra}
\nextlevel[5]
\mess{}{ra}{rb}
\mess{$(h_A)_A$}{ra}{rc}
\nextlevel[2]
\mess{$(h_B)_B$}{rb}{ra}
\nextlevel[2]
\mess{$(h_C)_C$}{rc}{ra}
\nextlevel[2]
\mess{}{ra}{rb}
\mess{$((h_A)_A, (h_B)_B, (h_C)_C)_A$}{ra}{rc}
\nextlevel[2]
\mess{$((h_A)_A, (h_B)_B, (h_C)_C)_B$}{rb}{ra}
\nextlevel[2]
\mess{$((h_A)_A, (h_B)_B, (h_C)_C)_C$}{rc}{ra}
\end{msc}
\label{fig:key_gen_example}
\caption{Example of distributed key-generation for 3 registrars}
\end{figure}

The protocol uses $O\left(n^2\right)$ messages and is vulnerable to an attack in
which an attacker simply does not participate in it. However in the large scope
of the system it's not an issue, because we assume that introducing a registrar
requires an organizational approval which is a nontrivial barrier of entry and
later any malicious nodes may be easily detected and removed from the
centralized lists should it become compromised.

\subsection{Node join protocol}

Every node in order to participate in the DHT network is required to have a node
certificate. Node certificate is a collection of certificates from registitration
nodes certyfying validity of node identification ([TODO tuple of DHT ID and
public key]). A certificate provided by registration nodes is a tuple
$\left(\text{node}_{ID}, \, \text{expiration date}, \, \text{signature}\right)$.
Certificate serves as a proof to other nodes that the key in the node ID has been
generated by that registration authority and that the registration authority
allowed this node to join the DHT.H

The system's threat model allows some registration authorities to be malicious
and potentially generate biased keys. The node join protocol prevents such nodes
from performing a successful attack. Given $m$ malicious nodes the node join
protocol at least $m+1$ certificates. The following scheme assumes that $m = 1$.

\begin{figure}
\begin{msc}{Node join protocol}
\setlength{\instdist}{5.5cm}
\setlength{\envinstdist}{3cm}
\declinst{client}{C}{Client}
\declinst{ra}{RA}{Registrar A}
\declinst{rb}{RB}{Registrar B}
\mess{Perform challenge protocol}{client}{ra}
\nextlevel[2]
\mess{Perform challenge protocol}{client}{rb}
\nextlevel
\action*{
\begin{minipage}{4cm}\centering
  Has received nonces $r_A, r_B$ during the challenge protocol.
\end{minipage}}{client}
\nextlevel[4]
\mess{$(r_A, r_B)$}{client}{ra}
\nextlevel[2]
\mess{$(r_A, r_B)$}{client}{rb}
\nextlevel[2]
\mess{$(r_A, r_B)_A$}{ra}{client}
\nextlevel[2]
\mess{$(r_A, r_B)_B$}{rb}{client}
\nextlevel[2]
\mess{$(r_A, r_B)_B$}{client}{ra}
\nextlevel[2]
\mess{$(r_A, r_B)_A$}{client}{rb}
\nextlevel[2]
\referencestart{r}{robust key-generation protocol}{ra}{rb}
\nextlevel[2]
\gate[r][b]{out}{rleft}
\mess{\parbox{4cm}{Node certificates \\ signed by A and B\\}}{rleft}{client}
\nextlevel[1]
\referenceend{r}
\end{msc}
\caption{Node's join protocol}
\end{figure}

[TODO certificate refresh]

\subsection{Certificate revocation system}

\subsection{SybilControl - Limiting number of sybil nodes with computational
puzzles}
\subsubsection{Goals}
  Verifiable random key generation in Kademlia is a defense against eclipse
  style attacks, which in Kademlia are much harder to launch than in other
  DHTs, such as Chord, due to nonstrict routing tables and redundant, iterative
  routing algorithm. [TODO citation]. However eclipse attack might not be
  necessary if attacker can perform a Sybil attack. Although this the scope of
  this attack will be limited by the centralized authority thanks to the use of
  computation puzzles, it just limits the rate of entry.

  The centralized authority allows for implementation of various anti-sybil
  check schemes, but it increases responsibilities of those nodes which we might
  want to keep as simple as possible. So in order to provide a defense against
  Sybil attack Ghoul incorporates SybilControl into its mechanisms.

\subsubsection{SybilControl mechanism}
  SybilControl is fully described in \cite{li12} here I will describe the core
  mechanism as it is used in Ghoul. 

  To limit the number of sybil nodes SybilControl requires that the nodes
  generate a cryptographic puzzle periodically and solve it. To verify that this
  puzzle is generated honestly it includes similarly generated puzzles prpagated
  from its neighbours. The puzzle string for node $A$ is $C_{A-new}$, who has
  received puzzles $C_{B_i}$ from its neighbours $B_{i}$ where:

  \begin{eqnarray*}
      R_{A-new} &=& B_1||C_{B_1}||\ldots||B_n||C_{B_n}||r_A||C_{A-old}\\
    C_{A-new} &=& H\left(R_{A-new}\right)
  \end{eqnarray*}

  $r_A$ is a random string generated by $A$. $H$ is a one-way function such as
  \texttt{SHA-2}. This is to ensure that $r_A$ is not specially prepared to make
  the puzzle solving easy.

  The solution to the puzzle is such a string $S$ that:

  \[ h = H\left(A||C_A||S\right)\]

  has at least $p$ last bits equal to zero.

  The entire puzzle state $P$ consists ofm: the solution $S_P$, the challenge
  $C_P$ and the record state $R_P$. Generated puzzle states are kept for some
  globally defined for verification purposes.

  \paragraph{Neighbour node verification}
  To verify that a neighbour node has done its proof-of-work it is asked to send
  its latest puzzle state $P$. Then the asking node checks whether this is a
  valid, solved puzzle state and 

  \paragraph{Non-neighbour node verification}
  To verify that a non-neighbour node has solved its puzzle in the recent past we
  need to check that our recent challenge has influenced its puzzle. To do that
  we find a path between us and the non-verified node. We query nodes in the
  path for puzzle state history. The node verification is then equivalent to
  finding out in these puzzle state history all puzzles that have influenced the
  puzzle of the node under verification. The node is verified iff a puzzle that
  we have produced is among them.

\subsubsection{SybilControl in Ghoul}
  SybilControl requires additional consideration as to how it integrates with
  Kademlia.

  First off the protocol loosely defines what constitutes a neighbour. An
  uncareful implementation of SybilControl in which every cryptographic puzzle
  propagation is included in the generation of the next puzzle may increase the
  size of the puzzle size considerably. This kind of behaviour may be used to
  slow down the DHT by gratuitous send of challenge updates to all nodes in the
  DHT. Li et al. do not take this into consideration, but this can be prevented
  by including only challenges received from nodes in the routing table and
  additionally limiting the number of included challenges to those that have
  been received in the last $2p$ seconds from the $i$ closests nodes. Where $p$
  and $i$ are globally defined constants.

  Secondly the protocol described in SybilControl has only local effects, that
  is a node that does not have an up to date puzzle will just not be included
  into the routing table. Since Ghoul uses a certification system with
  revocation it is possible for nodes to complain about others to registrars. If
  registrars find that given node is malicious or sybil they may globally revoke
  its certificate, significantly increasing the cost of a Sybil attack.

  [About additional advantages of Sybil control in Ghoul]
  [Problem with revocation spam]

\chapter{Ghoul implementation}
\label{ch:implementation}
In this chapter we present Ghoul's implementation.
We focus on general techniques used in Ghoul's development to produce high-quality code.
In section \ref{sec:build} we present build tools used and an example session with Ghoul.
In section \ref{sec:implementation} we show, as an example of our design principles, the interface of \texttt{KademliaRouting} - the core class in the protocol.

Ghoul is an open-source project.
An up-to-date Ghoul's code is available for public audience at \url{https://github.com/gregorias/ghoul}.

\section{Build system and running}
\label{sec:build}

\subsection{Gradle}
Currently Ghoul's code contains around 8000 lines of code in 126 Java files
(excluding dfuntest's project).
To maintain this amount of code we use Gradle
\footnote{\url{https://gradle.org/}} build automation framework.
In Gradle, developers write configuration scripts called \texttt{build.gradle}
in which they define code's properties --- e.g. name of the project, project's
dependencies; and build tasks.
Those configuration scripts are written in an imperative, Java-like, dynamically
typed language.
This language allows for expressive and succinct definition of build tasks,
which is an advantage over more traditional build framework - Maven
\footnote{\url{https://maven.apache.org/}}.
Next, Gradle performs, using provided information, typical tasks in Java's build
ecosystem: compilation, packaging, dependency resolution, unit testing, software
validation etc.
Gradle allows developers to download and build Ghoul in two simple steps:
download the software from GitHub and run build task in Gradle (shown in figure
\ref{fig:ghoul_build_process})

\begin{figure}[tb]
\begin{verbatim}
> git clone https://github.com/gregorias/ghoul
...

> cd ghoul

> ./gradlew build
:assemble UP-TO-DATE
:check UP-TO-DATE
:build UP-TO-DATE
:ghoul-core:compileJava
:ghoul-core:processResources UP-TO-DATE
:ghoul-core:classes
:ghoul-core:jar
:ghoul-core:javadoc
...
:ghoul-dfuntest:pmdMain
:ghoul-dfuntest:pmdTest UP-TO-DATE
:ghoul-dfuntest:test UP-TO-DATE
:ghoul-dfuntest:check
:ghoul-dfuntest:build

BUILD SUCCESSFUL

Total time: 31.617 secs
\end{verbatim}
\caption{Ghoul's build process}
\label{fig:ghoul_build_process}
\end{figure}

We have added additional automatic code quality checking tools to Ghoul's automatic build:

\begin{description}
  \item{\textbf{jUnit}\footnote{\url{http://junit.org}}} 
    jUnit is the framework for writing repeatable unit tests for Java code.
  \item{\textbf{JaCoCo}\footnote{\url{www.eclemma.org/jacoco/}}}
    JaCoCo is a code coverage library which automatically analyzes code executed by jUnit tests and produces an HTML or XML report summarizing code coverage.

  \item{\textbf{PMD}\footnote{\url{http://pmd.sourceforge.net}}}
    PMD is a source code analyzer.
    It finds common programming flaws like unused variables, empty catch blocks, unnecessary object creation etc.
  \item{\textbf{Findbugs}\footnote{\url{http://findbugs.sourceforge.net/}}}
    Findbugs is a static analysis tools which looks for bugs in Java's code.
    It can detect such bugs are code smells as: unused variables, null value dereference, use of \texttt{AtomicBoolean} for comparison etc.
    
\end{description}

We haven't included dfuntest's test into the automatic build, because this is a long task.
However dfuntest's testing suite is always run manually to confirm code's correctness.

\subsection{Usage}

Although Ghoul is mainly designed to be used as a library, it is also possible to run it as a standalone application.
In standalone mode Ghoul additionally exposes an RPC-over-HTTP interface over which it is possible to query and manipulate the DHT node (shown in table \ref{tab:http_rpc}).

An example session which runs 2 registrars and 3 DHT nodes is shown in figures 
\ref{fig:ghoul_manual_run} and \ref{fig:ghoul_manipulation}.
In the session we use curl to send RPC-over-HTTP queries to DHT nodes.
First we start registrars and then DHT nodes.
Afterwards we get routing tables of the second DHT node.
Then we put \texttt{DATA} string under the key \texttt{100}.
Later we ask the third node get all replicas of the data under key \texttt{100}.
The data returned is the \texttt{DATA} string encoded with BASE64 encoding.

\begin{figure}[tbp]
\begin{verbatim}
> java -cp ghoul-core-0.1.jar:lib/* me.gregorias.ghoul.interfaces.RegistrarMain\
  registrar0.xml > /dev/null &

> java -cp ghoul-core-0.1.jar:lib/* me.gregorias.ghoul.interfaces.RegistrarMain\
  registrar1.xml > /dev/null &

> java -cp ghoul-core-0.1.jar:lib/* me.gregorias.ghoul.interfaces.Main \ 
  kademlia0.xml >/dev/null &

> java -cp ghoul-core-0.1.jar:lib/* me.gregorias.ghoul.interfaces.Main \ 
  kademlia1.xml >/dev/null &

> java -cp ghoul-core-0.1.jar:lib/* me.gregorias.ghoul.interfaces.Main \ 
  kademlia2.xml >/dev/null &

> curl -X POST 127.0.0.1:10000/start

> curl -X POST 127.0.0.1:10001/start

> curl -X POST 127.0.0.1:10002/start
\end{verbatim}
\caption{Ghoul's starting commands}
\label{fig:ghoul_manual_run}
\end{figure}

\begin{figure}[tbp]
\begin{verbatim}
> curl -X GET 127.0.0.1:10001/get_routing_table
{"nodeInfo":[{"inetAddress":"localhost","port":9000,
    "key":"15624f385826cae60bef58da5b53b4dc1cf070bd"},
{"inetAddress":"localhost","port":9002,
    "key":"2d1386c9c601676189faa7790b60dbeda3ed3e9b"}]}

> curl -X POST --header "Content-Type:application/octet-stream" \ 
--data-binary DATA 127.0.0.1:10000/put/100

> curl -X GET 127.0.0.1:10002/get/100
["REFUQQ==","REFUQQ==","REFUQQ=="]
\end{verbatim}
\caption{Ghoul's status query and DHT manipulation}
\label{fig:ghoul_manipulation}
\end{figure}

\section{Implementation}
\label{sec:implementation}

The core interface of the protocol is \texttt{KademliaRouting} (shown in figure \ref{fig:routing_interface}).
Implementations of this interface should implement the Kademlia protocol with Ghoul extensions.
One of the focus of Ghoul's code is modularizabilty and \texttt{KademliaRouting} is an example of that focus.
First, \texttt{KademliaRouting} is an interface, although there is only one implementation of it in the public part of the library's code.
However, because it is an interface it is possible to easily mock it in unit-tests.
\texttt{KademliaRouting} has another implementations in unit-testing code, e.g. \texttt{StaticKademliaRouting}, which has static routing table.
This allows creation of proper, isolated unit-tests that test non-trivial scenarios and do not depend on complex, error-prone dependencies.
Second, \texttt{KademliaRouting} does not implement storage functionality.
Storage is done by \texttt{KademliaStore} class.
Such division of responsibilities is also an example of Single Responsibility
Principle - one class should have only one responsibility.

\begin{figure}[tbp]
\begin{lstlisting}
public interface KademliaRouting {
  /**
   * Find size number of nodes closest to given {@link Key}.
   *
   * @param key key to look up
   * @param size number of nodes to find
   * @return up to size found nodes
   */
  Collection<NodeInfo> findClosestNodes(Key key, int size) throws InterruptedException,
      KademliaException;

  /**
   * @return hosts present in the local routing table.
   */
  Collection<NodeInfo> getFlatRoutingTable();

  /**
   * @return Key representing this peer
   */
  Key getLocalKey();

  /**
   * @return is kademlia running.
   */
  boolean isRunning();

  /**
   * Registers neighbour listener which will receive notifications about newly added nodes.
   *
   * @param listener listener to register
   */
  void registerNeighbourListener(NeighbourListener listener);

  /**
   * Unregisters neighbour listener if any is present.
   */
  void unregisterNeighbourListener();

  /**
   * Connect and initialize this peer.
   */
  void start() throws KademliaException;

  /**
   * Disconnects peer from network.
   */
  void stop() throws KademliaException;
}
\end{lstlisting}
\caption{\texttt{KademliaRouting} interface}
\label{fig:routing_interface}
\end{figure}

{\texttt{KademliaRoutingImpl}'s constructor} is shown in figure \ref{fig:routing_constr_header}.
This constructor shows dependencies of \texttt{KademliaRoutingImpl}.
Functions such as: discovering current internet address
(\texttt{NetworkAddressDiscovery}), sending messages to given network address
(\texttt{MessageSender}), background thread execution
(\texttt{ScheduledExecutorService}), are realized by interfaces.
This allows custom definition of those functionalities by the user.
This is realization of yet another principle called dependency inversion - dependencies of classes should be provided to the classes not created by it.
Again, the biggest strength of such approach is mostly visible in unit tests.
Since network code is abstracted by interfaces, unit-tests may use fake implementations which transmit messages in memory and inject failures arbitrarily.
Additionally, users of the Ghoul library may, for example, connect \texttt{KademliaRoutingImpl} with their own network code.
Large number of parameters is sometimes considered a code smell, but dependency inversion is a more important consideration in this case.
Additionally a builder design pattern is used for creating instances of \texttt{KademliaRouting}.

\begin{figure}[tbp]
\begin{lstlisting}
KademliaRoutingImpl(Key localKey,
                    NetworkAddressDiscovery networkAddressDiscovery,
                    MessageSender sender,
                    ListeningService listeningService,
                    int bucketSize,
                    int alpha,
                    Collection<NodeInfo> initialKnownPeers,
                    long messageTimeout,
                    TimeUnit messageTimeoutUnit,
                    long heartBeatDelay,
                    TimeUnit heartBeatDelayUnit,
                    PersonalCertificateManager personalCertificateManager,
                    CertificateStorage certificateStorage,
                    PrivateKey personalPrivateKey,
                    ScheduledExecutorService scheduledExecutor,
                    Random random);
\end{lstlisting}
\caption{\texttt{KademliaRoutingImpl}'s constructor}
\label{fig:routing_constr_header}
\end{figure}

\chapter{Dfuntest - Streamlined functional testing framework for distributed applications}
\section{Chapter introduction and overview}
Good practices of software engineering agree \cite{mccon04} that quality code has to be tested and testable.
While many single-application projects rely only on extensive unit-testing frameworks, a distributed application requires more encompassing tools.

Simulation is the most common method to test parallel and distributed algorithms and systems in research.
With right tools (such as SimGrid\cite{cas08}), the simulation environment is realistic, and the whole experiment reproducible.
However, a simulation experiment rarely covers the whole distributed system, from network packets to high level functionalities.
Instead, only a certain, isolated fragment is tested.
Although the simulations confirm the efficiency of an algorithm, the entire application requires a lot more functionalities which are not covered by the simulation, such as the network code, resource management etc.

For the project to be accepted by a wider audience the code needs to be of sufficient quality.
However, performing and reproducing experiments with a distributed application is tedious.
Not only do we have to code the application, but also we need to configure the experimental platform, deploy the application; then, after performing a particular test scenario, gather the results and analyze them.
Developers commonly write ad-hoc scripts to automate many of these tasks; however, these scripts are repetitive, full of boilerplate, and do not add any new core functionality.
Moreover, such scripts are hard to maintain and port between e.g. user credentials, or experimental platforms (e.g. a local cluster or Planet-lab).

We developed dfuntest to test our implementation of Kadmlia, but the project's ambition is to facilitate the process of testing distributed applications in general. 
Dfuntest not only defines a flexible testing pattern, but also automates common tasks such as preparing remote environments, deploying the code, running the application and test cases, generating a report, and downloading log files.

Dfuntest is written in Java but the tested application may be written in any language since dfuntest tests the application by querying the application's external interface.
Having the testing library and the tests in a statically-typed language helps to avoid common bugs, which would be only visible at the runtime in a dynamically-typed language. 
Static typing also provides a way of documentation.

The ultimate goal of dfuntest is to become an equivalent of jUnit for distributed applications.
A tool that allows developers to focus on the logic in tests, rather than diluting in repetitive code and complex, partially-automated procedures.
A tool that facilitates switching parameters, settings, even platforms.
Finally, a tool that enables others to easily reproduce reported results and verify application's usefulness in their environment.

In this chapter dfuntest is presented.
First, we define a design pattern for testing distributed applications (Section~\ref{sec:dfuntest-design}).
Second, we present (Section~\ref{sec:dfunt-impl}) a concrete implementation of the proposed pattern --- the dfuntest framework.
We evaluate the framework by testing our DHT implementation (Section~\ref{sec:exampl-test-kademl}): in particular, we show how to verify that the implementation maintains the correct DHT topology.
Dfuntest is available as a separate project at \url{https://github.com/gregorias/dfuntest}.

Dfuntest was initially built for our bare Kademlia implementation.
Ghoul extends Kademlia with additional background security protocols, therefore distributed tests for Kademlia are also tests for Ghoul (with small adjusments in environment preparation).
For this historical reason the name Kademlia is used in examples in section~\ref{sec:exampl-test-kademl}.

\section{Dfuntest design}\label{sec:dfuntest-design}

\subsection{Abstracting tests of distributed applications}
One of the key features of testing frameworks are the structural abstractions of
the testing process. These abstractions are a result of a delicate equilibrium.
They must be general enough not to limit tester's expressiveness; but they must
be feasible to automate by the library code. A distributed testing framework
is no exception.

We will use the following vocabulary when describing the architecture. 
The framework runs on a single host called the \emph{testmaster}.
The application/system tested by the framework is the \emph{tested application}.
This application is composed of many \emph{instances};
each instance runs on an \emph{environment}---a remote host, or a separate directory on the testmaster.

We recognize the following phases when running a single test of a distributed
application:

\begin{enumerate}
\item \textbf{Test configuration} The tester decides which scenarios to run, on what
  hosts, and with which parameters.
\item \textbf{Environment preparation} The application and the test data are deployed on the target environments.
\item \textbf{Testing} The tested application runs according to a pre-defined \emph{scenario}; the test checks assertions on the application's state.
\item \textbf{Report generation} A report includes the test
  result and supplementary information for debugging, such as logs.
\item \textbf{Clean up} Generated artifacts (and perhaps also the environment) are deleted from remote hosts.
\end{enumerate}

A feature specific to distributed tests is the point of control of a test---that
describes how the testing scenario is executed. The point of control has the
following design axes:

\begin{description}
  \item{\textbf{Decentralization}} The degree by which the code describing the testing scenario is itself distributed.

    Some testing scenarios are inherently distributed, e.g., checking whether
    neighbors of a host respond to a periodic ping.  However, centralized
    control is more flexible, as it is easier to verify global properties of the
    application (e.g.: whether the induced graph is connected, see
    Section~\ref{sec:exampl-test-kademl}).  A scenario with distributed control
    can be centralized by exposing the requested behavior in an external
    interface of the tested application. In contrast, decentralization of a centralized
    test is more difficult.
    We refer to~\cite{ulr99,hie08,hie12} for further discussion.

    Centralized control introduces a single point of responsibility.
    This is not a true weakness, since it allows for more accurate detection and reasoning about errors.
    More importantly, a centralized control is less scalable.
    Lesser scalability might prevent the framework from effective testing of larger networks.

  \item{\textbf{Interactivity}} How much does the testing control interacts with
    the tested application over the course of the test.
    For example an interactive test may analyze workload and responsibilities of
    individual nodes in the tested application and try to put additional work to
    nodes whose failure may break the tested application.
\end{description}


\subsection{Architecture of Dfuntest}

\begin{figure}[tbp]
  \centering
  \def\svgwidth{\columnwidth}
  \scriptsize {
  \input{dfuntest_deployment.pdf_tex}
}
\caption{Deployment architecture}
\label{fig:4_deployment_architecture}.
\end{figure}

Dfuntest assumes a centralized control of a test. A single master
host --- the testmaster --- executes all phases of the test.
Tests scenarios can be highly interactive.
They are limited only by the interface provided by the tested application.
The deployment architecture of dfuntest testing process is shown on Figure \ref{fig:4_deployment_architecture}.

We describe the architecture of dfuntest by tracing the flow of a single test (as
described in the previous section); to motivate certain design decisions we
start our description from the \emph{testing} phase.

The testing phase consists of executing a certain scenario with actors
corresponding to applications running on (remote) environments.
This scenario is a \texttt{TestScript}; it is defined by the tester. The
scenario needs actors which are (remote) instances of the application --- or, more
accurately, proxies to external interfaces of the instances of the application that run on (remote) 
environments.
This proxy is represented by an \texttt{App} class.

To prepare the test environment on remote hosts, our framework uses standard OS
tools to copy, upload, or download files, traverse directories, run processes
etc.  Dfuntest abstracts from concrete implementations of these tools by a proxy
class \texttt{Environment}.
An abstract \texttt{Environment} enables dfuntest to abstract from where the
environment is located or how it is accessed.
This separation of \texttt{Environment} from \texttt{App} gives greater
flexibility and allows testing an application in various environments.
One frequent use-case is a test failing on a remote environment---by changing
the remote environment to a local one (i.e., the whole system runs as multiple
processes on a single host), debugging is faster and more accessible.

The process of deploying an instance of the tested application to an
environment (e.g., choosing which test data to put on which hosts)
depends strongly on the particular user application. This functionality is
described by the tester and encompassed in \texttt{EnvironmentPreparator}.

A test should result in an artifact documenting its results and logs to debug
possible failures. In dfuntest, the \texttt{TestScript} is responsible for this
functionality.

After tests are finished we may want to clean up environments and download test's artificats.
These are additional responsibilities of \texttt{EnvironmentPreparator}.

The class architecture of the process described above is shown on Figure
\ref{fig:4_class_architecture}.

\begin{figure}[tbp]
  \centering
  \def\svgwidth{\columnwidth}
  \scriptsize {
  \input{dfuntest_bw2.pdf_tex}
}
\caption{Class architecture}
\label{fig:4_class_architecture}
\end{figure}

\section{Dfuntest implementation}\label{sec:dfunt-impl}

In this section, we describe how dfuntest maps the abstractions sketched in the previous section to concrete code. Dfuntest defines a number of interfaces and provides reusable tools (such as concrete \texttt{Environment}s used for interacting with remote hosts) to stitch a coherent testing framework.

\paragraph{\texttt{Environment}} interface (Figure~\ref{fig:env_interface}) represent a proxy
object that performs typical shell operations on an environment.
Dfuntest provides two implementations of an \texttt{Environment}:
a \texttt{LocalEnvironment} and an \texttt{SSHEnvironment}.
In a local test, an application is deployed to multiple directories of the testmaster; the \texttt{LocalEnvironment} acts on the provided directory.
An \texttt{SSHEnvironment} connects to a remote host and translates method calls to SSH functions, e.g. \texttt{copy} to \texttt{scp}.

A tester may want to add new functions to \texttt{Environment}. For this reason, we defined other dfuntest interfaces as generics, taking a subclass of \texttt{Environment} as a parameter. 

\begin{figure}[tbp]
\begin{lstlisting}
public interface Environment {
  void copyFilesFromLocalDisk(Path srcPath, String destRelPath)
    throws IOException;
  void copyFilesToLocalDisk(String srcRelPath, Path destPath)
    throws IOException;
  String getHostname();
  int getId();
  String getName();
  Object getProperty(String key) throws NoSuchElementException;
  RemoteProcess runCommand(List<String> command)
    throws InterruptedException, IOException;
  RemoteProcess runCommandAsynchronously(List<String> command)
    throws IOException;
  void removeFile(String relPath) throws InterruptedException, IOException;
  void setProperty(String key, Object value);
}
\end{lstlisting}
\caption{\texttt{Environment} interface}
\label{fig:env_interface}
\end{figure}

\paragraph{\texttt{EnvironmentPreparator}} (Figure \ref{fig:envprepint}) defines the environmental dependencies between the application and its environment. The preparator prepares the environment, collects any output (including logs), and cleans the environment. The tester implements the preparator, as this process is specific to a concrete application.

Some applications depend on many external libraries, or datasets; copying these
files to remote hosts takes time.
To speed-up environment preparation for subsequent tests in a test suite, we
split the preparation process into two methods:
\texttt{prepare} assumes an empty environment, and thus copies all dependencies;
while \texttt{restore} assumes that all read-only files have been loaded.

\begin{figure}[tbp]
\begin{lstlisting}
public interface EnvironmentPreparator<EnvironmentT extends Environment> {
  // Prepare the environment from scratch.
  void prepare(Collection<EnvironmentT> envs) throws IOException;
  // Restore the environment after cleanOutput
  void restore(Collection<EnvironmentT> envs) throws IOException;
  // Download output and log files to local destPath
  void collectOutput(Collection<EnvironmentT> envs, Path destPath);
  // Remove all files generated by the application
  void cleanOutput(Collection<EnvironmentT> envs);
  // Restore the environment to clean state
  void cleanAll(Collection<EnvironmentT> envs);
}
\end{lstlisting}
\caption{\texttt{EnvironmentPreparator} interface}
\label{fig:envprepint}
\end{figure}

\paragraph{\texttt{App}}
The App interface is a proxy translating Java method calls to RPC invocations to a concrete instance of the tested application (running on a (remote) environment).
A tester should subclass \texttt{App} and add methods which correspond to the external RPC interface of their application.

\paragraph{\texttt{TestScript}}, the interface for testing scenarios, is the one
that will be subclassed the most and that is also the simplest. It implements
only one method \texttt{run} which executes the testing scenario, checks
assertions, and returns a report.

\paragraph{\texttt{EnvironmentFactory}, \texttt{AppFactory}} Since
\texttt{Environment} and \texttt{App} are meant to be subclassed, dfuntest uses
the Factory pattern to hide specific implementation from classes that do not require
it, like \texttt{TestRunner}.

\paragraph{\texttt{TestRunner}} A runner is an object which takes all previous
classes as dependencies, including a collection of \texttt{TestScript}s to run,
and runs the entire testing pipeline. It is a dfuntest equivalent of a
\texttt{Runner} in jUnit.
\texttt{TestRunner} uses the \texttt{EnvironmentFactory} to
create and prepare remote environments. Then, for each test, it uses the
\texttt{AppFactory} to create instances of \texttt{App}s (which in turn start
the remote instances of application). Once \texttt{App}s are created,
\texttt{TestRunner} runs \texttt{TestScript}s, collecting logs and cleaning
environments in-between. Finally it produces a report directory.


\section{Testing Ghoul DHT}\label{sec:exampl-test-kademl}
In this section we will present how we have prepared dfuntests for Ghoul.

The core Kademlia submodule uses UDP messages for internal communication between instances.
The application exposes an external RPC over HTTP.
This interface allows the user to control and to query Ghoul for typical DHT operations (put/get).
We show a subset of available external RPC in Table~\ref{tab:http_rpc}.

\begin{table}[tbp]
  \begin{tabular}{|c|p{3cm}|p{3.6cm}|p{3.4cm}|}
    \hline
    HTTP-RPC & Arguments & Response & Description \\
    \hline
    \texttt{START} & None & Plain HTTP & Start Kademlia service.\\
    \hline
    \texttt{STOP} & None & Plain HTTP  & Stop Kademlia service.\\
    \hline
    \texttt{SHUT\_DOWN} & None & Plain HTTP &
      Shutdown the entire application after returning this call.\\
    \hline
    \texttt{FIND\_NODES} & Kademlia key & JSON list of key-ip pairs &
      Find Kademlia's neighbors close to given key.\\
    \hline
    \texttt{GET\_ROUTING\_TABLE} & None & JSON list of key-ip pairs & Return
      Kademlia node's neighbors present in its routing table.\\
    \hline
    \texttt{GET\_KEY} & None & String representing the key & Return callee's
      Kademlia ke.y\\
    \hline
    \texttt{GET} & Kademlia key  & JSON list of binary data & Get instances of
    data stored under given key. \\
    \hline
    \texttt{PUT} & Kademlia key and binary data  & Plain HTTP & Store given
    key-data pair in Kademlia. \\
    \hline
  \end{tabular}
  \caption{RPC-over-HTTP interface of Ghoul application}
  \label{tab:http_rpc}
\end{table}

\subsection{Preparation}
We need to provide to dfuntest information on how to start and use our
application. This is done only once and does not need to be rewritten as long as
the Ghoul API and its requirements remain the same.

\paragraph{\texttt{App}} 
\texttt{KademliaApp} extends the proxy \texttt{App} class with Ghoul's interface methods.
\texttt{KademliaApp}'s main responsibility is to translate Java method calls into the RPC-over-HTTP interface of the Ghoul application.
An example code of those methods is shown in \ref{fig:app_example}.
\texttt{KademliaApp} is parametrized by the base \texttt{Environment}, as we use only standard operations.
To construct, \texttt{KademliaApp} requires the URI address of the external
application's interface; and the environment on which it works.

\begin{figure}[tbp]
\begin{lstlisting}
public synchronized void startUp() throws IOException {
  List<String> runCommand = new LinkedList<>();
  runCommand.add(mJavaCommand);
  runCommand.add("-Dorg.slf4j.simpleLogger.logFile=" + LOG_FILE);
  runCommand.add("-Dorg.slf4j.simpleLogger.defaultLogLevel=trace");
  runCommand.add("-cp");
  runCommand.add("lib/*:Kademlia.jar");
  runCommand.add("me.gregorias.Kademlia.interfaces.Main");
  runCommand.add("Kademlia.xml");
  mProcess = mKademliaEnv.runCommandAsynchronously(runCommand);
}

public Collection<NodeInfo> findNodes(Key key) throws IOException {
  Client client = ClientBuilder.newClient();
  WebTarget target = client.target(mUri).path("find_nodes/" + key.toInt());

  NodeInfoCollectionBean beanColl;
  try {
    beanColl = target.request(MediaType.APPLICATION_JSON_TYPE)
        .get(NodeInfoCollectionBean.class);
  } catch (ProcessingException e) {
    throw new IOException("Could not find node.", e);
  }
  NodeInfoBean[] beans = beanColl.getNodeInfo();
  Collection<NodeInfo> infos = new LinkedList<>();
  for (NodeInfoBean bean : beans) {
    infos.add(bean.toNodeInfo());
  }
  return infos;
}
\end{lstlisting}
\caption{\texttt{KademliaApp}: start a remote Ghoul instance; and use
HTTP-RPC to find neighboring nodes of the instance.}
\label{fig:app_example}
\end{figure}

\paragraph{\texttt{KademliaEnvironmentPreparator}} (Figure~\ref{fig:prepare})
\texttt{KademliaEnvironmentPreparator} copies Kademlia's dependency jar files and configuration
files to to the target environment.

\texttt{KademliaEnvironmentPreparator} uses methods of \texttt{Environment} to deploy the application.
graph;
\begin{figure}[tbp]
\begin{lstlisting}
public void prepare(Collection<Environment> envs)
    throws IOException {
  Collection<Environment> preparedEnvs = new LinkedList<>();
  Environment firstEnv = findFirstEnvironment(envs);
  for (Environment env : envs) {
    XMLConfiguration xmlConfig = prepareXMLConfiguration(env, firstEnv);
    try {
      xmlConfig.save(XML_CONFIG_FILENAME);
      Path localConfigPath = FileSystems.getDefault().getPath(LOCAL_CONFIG_PATH).toAbsolutePath();
      env.copyFilesFromLocalDisk(localConfigPath, ".");
      env.copyFilesFromLocalDisk(LOCAL_JAR_PATH.toAbsolutePath(), ".");
      env.copyFilesFromLocalDisk(LOCAL_LIBS_PATH.toAbsolutePath(), ".");
      preparedEnvs.add(env);
    } catch (ConfigurationException | IOException e) {
      clean(preparedEnvs);
      throw new IOException(e);
    }
  }
}
\end{lstlisting}
\caption{\texttt{KademliaEnvironmentPreparator} (fragment): prepare remote environments by copying jars and configuration files. Methods \texttt{restore}, \texttt{collectOutput}, \texttt{cleanOutput} and \texttt{cleanAll} are similar.}
\label{fig:prepare}
\end{figure}


\paragraph{Main and \texttt{KademliaAppFactory}} To glue those we also need to
define a simple factory which instantiates \texttt{KademliaApp}s given prepared
\texttt{Environments}. Also the entry point to dfuntest application - the main
method - needs to be defined.

\subsection{Test script}

We show how to check whether the topology graph induced by Kademlia routing
tables is connected (consistent).

To define the testing scenario, the tester writes the \texttt{run} method defined by the
\texttt{TestScript} interface (Figure~\ref{fig:core}).

\begin{figure}[tbp]
\begin{lstlisting}
public TestResult run(Collection<KademliaApp> apps) {
  try {
    startUpApps(apps); // Call startUp method of each KademliaApp.
    Thread.sleep(START_UP_DELAY_UNIT.toMillis(START_UP_DELAY));
  } catch (IOException e) {
    return new TestResult(Type.FAILURE,
        "Could not start up applications.", e);
  }
  try {
    startKademlias(apps); // Send start HTTP-RPC to each interface
  } catch (KademliaException | IOException e) {
    shutDownApps(apps);
    return new TestResult(Type.FAILURE, "Could not start Kademlias.", e);
  }
  mResult = new TestResult(Type.SUCCESS,
      "Connection graph was consistent the entire time.");
  scheduleCheckerToRunPeriodically(new ConsistencyChecker(apps))
  waitTillCheckerFinished();
  stopKademlias(apps);
  shutDownApps(apps);
  return mResult;
}

private class ConsistencyChecker implements Runnable {
  @Override
  public void run() {
    Map<Key, Collection<Key>> graph;
    try {
      // Call GET_ROUTING_TABLE RPC of each node
      graph = getConnectionGraph(mApps);
      ConsistencyResult result = checkConsistency(graph);
      if (result.getType() == ConsistencyResult.Type.INCONSISTENT) {
        mResult = new TestResult(Type.FAILURE,
            "Graph is not consistent starting from: " + result.getStartVert()
            + " could only reach " + result.getReachableVerts().size());
        shutDown();
        return;
      }
    } catch (IOException e) {
      mResult = new TestResult(Type.FAILURE,
          "Could not get connection graph: " + e + ".");
      shutDown();
      return;
    }
    --mCheckCount;
    if (mCheckCount == 0) {
      shutDown();
    }
  }
  ...
}
\end{lstlisting}
\caption{\texttt{KadmliaConsistencyTestScript} (fragment) periodically checks consistency of the induced Kademlia graph.}
\label{fig:core}
\end{figure}

\subsection{Usage}

We configured Ghoul build system to create a separate .jar package with the above dfuntest.
The tests are executed as a standard Java application.
The main method expects an XML configuration file as its argument.
This configuration file contains all pertinent parameters for setting up
environments and controlling test execution. For example to start a test suite on
7 remote hosts we provide an XML configuration file with following information:

\begin{verbatim}
<SSHEnvironmentFactory>
  <hosts>
    roti.mimuw.edu.pl,
    prata.mimuw.edu.pl,
    planetlab2.wiwi.hu-berlin.de,
    planetlab2.s3.kth.se,
    planetlab1.u-strasbg.fr,
    planetlab01.tkn.tu-berlin.de,
    pl1.uni-rostock.de
  </hosts>
  <username>
    mimuw_user
  </username>
  <privateKeyPath>
    /home/mimuw_user/.ssh/id_rsa
  </privateKeyPath>
</SSHEnvironmentFactory>
\end{verbatim}

The rest is handled by the dfuntest application and at the end a human-readable
report directory is produced. This report directory contains a separate
directory for each executed TestScript test with logs and summary report.
Additionally there's a summary for all the executed tests
(Figure~\ref{fig:sumrep}).

\begin{figure}[tbp]
\begin{verbatim}
[FAILURE] KademliaConsistencyCheckTestScript
[SUCCESS] KademliaPingTestScript
[SUCCESS] KademliaDataReplicationTestScript
\end{verbatim}
\caption{A summary report for the test in which one \texttt{TestScript} has failed}
\label{fig:sumrep}
\end{figure}

To inject a failure into Ghoul, we used a too small bucket size (1, compared with a usual value of 20).
Such small bucket should make the graph inconsistent, because node finding messages will have too little diversity.
The generated summary (Figure~\ref{fig:conrep}) clearly points to a misbehaving node (7). 
This points the tester where to look to the source of this error.

\begin{figure}[tbp]
\begin{verbatim}
[FAILURE] Found inconsistent graph.
The connection graph was:
3: [2, 0, 4]
0: [1, 2, 4]
1: [0, 2, 4]
6: [4, 0]
7: [4, 0]
4: [5, 6, 0]
5: [4, 6, 0]
2: [3, 0, 4]

Its strongly connected components are:
[7]
[3, 2, 0, 1, 6, 4, 5]
\end{verbatim}
\caption{A report generated after KademliaConsistencyCheckTestScript fails.}
\label{fig:conrep}
\end{figure}

In case an error requires a debugging process the developer may easily change
the testing environment and test parameters by changing the configuration file
to facilitate the bug triage.

\section{Related work}
The reader may suspect by now that the author might suffer from the "Not Invented Here" syndrome \footnote{\url{https://en.wikipedia.org/wiki/Not\_invented\_here}} (the author certainly asks himself this question from time to time).
After all, secure DHT and distributed testing of peer-to-peer applications should be solved problems.
Although several frameworks for testing distributed applications have been proposed, we found none that would have dfuntest scope and we would be able to use.
In this section we review research articles as well as available software that falls into dfuntest's scope

\cite{ulr99}~proposes a decentralized testing architecture and presents a tool for distributed monitoring and assessment of test events. This tool does not facilitate deployment automation.
In~\cite{tsa03}, a test scenario defined in an XML file uses tested application's external SOAP interface. 
While dfuntest also uses external interface, we envision that this interface is enriched for particular tests (new methods  added); moreover, scenarios as Java methods enable greater expressiveness.
In~\cite{hug04}, the code of the tested application is modified using aspects (thus, the framework tests only Java code). 
Given examples focus on monitoring rather than testing---tests can verify how many nodes are, e.g., executing a method.
\cite{de10}~uses annotations, which again limits the applicability of the framework to Java applications. 
The scenarios are defined in a pseudo-language that, compared to dfuntest, might increase readability, but also reduce expressiveness.
The framework is more distributed compared to dfuntest, as proxy objects (similar to our \texttt{App}) run on remote hosts. 
Remote proxies reduce the need for an external interface; 
however, dfuntest centralization helps to check assertions on whole state of the system.
\cite{tor10}~focuses on methods of isolating submodules by emulating some of the components---dfuntest tests the whole distributed application. 

To our best knowledge, frameworks described above are not publicly available. In addition to described differences, they do not abstract the remote environment (dfuntest's \texttt{Environment} and \texttt{EnvironmentPreparator}), thus they do not facilitate deployment, nor porting tests between user credentials or testing infrastructures.

We continue with the available software for distributed testing.
The Software Testing Automation Framework (STAF) \footnote{\url{http://staf.sourceforge.net}} is an open source
project that creates cross-platform, distributed software test environments. It
uses services to provide an uniform interface to environment's resources, such
as file system, process management etc. From an architectural point of view its
services correspond to \texttt{Environment} abstraction layer in dfuntest.
If needed, it is possible to extend \texttt{Environment} interface with STAF facilities and provide its features.

SmartBear TestComplete \footnote{\url{http://smartbear.com/product/testcomplete/overview}} proprietary product has distributed testing functionality.
The framework allows definition of arbitrary environments and runs jobs sequentially.
SmartBear does not provide any particular mechanism for running a testing scenario or generating a testing report.
Additionally the software requires that the environment has the TestComplete software installed and running and the application uses TestComplete bindings.

Robot Framework \footnote{\url{http://robotframework.org}} is a generic test automation framework for acceptance testing and acceptance test-driven development.
Users can define their testing scenarios in a high-level language resembling natural language.
Robot Framework then automates running and generating a testing report.
Robot does not provide any mechanisms for distributed test control and flexible distributed environment preparation.

Neither of those libraries covers the entire scope of dfuntest framework. What
all of them lack is the ability to define complex testing scenarios
programatically, which is provided by script and app abstraction layers.

\section{Conclusion and Future Work}

We have presented the dfuntest's design pattern for writing distributed tests with centralized control.
Dfuntest offers a coherent and expressive abstraction for distributed testing.
This abstraction allows for clean automation of the testing process which in turn also gives greater control of the real-world testing environment.

The main goal of dfuntest are jUnit-like acceptance tests: set up environments,
run the tested application, check some assertions on the state. However, thanks
to flexible test scripts, we are currently using dfuntest also for performance
evaluation of various subsystems of
nebulostore\footnote{\url{http://nebulostore.org}}, which previously required ad-hoc deployment scripts.

In the immediate future we plan to extend dfuntest library with ability to run tests of heterogeneous applications in heterogeneous environmnents---e.g., a client-server application where capabilities of environments are different.
We also plan better mechanisms for coping with failing environments: a failure of remote host, independent from the application. Currently such failure can only be detected with ssh queries.

\chapter{Experimental evaluation}
\label{ch:evaluation}

We use dfuntest to test and verify correctness of our implementation
(Section~\ref{sec:exampl-test-kademl}): in particular, we show how to verify
that the implementation maintains the correct DHT topology.

Dfuntest was initially built for our bare Kademlia implementation.
Ghoul extends Kademlia with additional background security protocols, therefore distributed tests for Kademlia are also tests for Ghoul (with small adjusments in environment preparation).
For this historical reason the name Kademlia is used in examples in
Section~\ref{sec:exampl-test-kademl}.
\section{Testing Ghoul DHT with dfuntest}
\label{sec:exampl-test-kademl}
In this section we will present how we have prepared dfuntests for Ghoul.

\subsection{Preparation}
We need to provide to dfuntest information on how to start and use our
application. This is done only once and does not need to be rewritten as long as
the Ghoul API and its requirements remain the same.

\paragraph{\texttt{App}} 
\texttt{KademliaApp} extends the proxy \texttt{App} class with Ghoul's interface methods.
\texttt{KademliaApp}'s main responsibility is to translate Java method calls into the RPC-over-HTTP interface of the Ghoul application.
An example code of those methods is shown in \ref{fig:app_example}.
\texttt{KademliaApp} is parametrized by the base \texttt{Environment}, as we use only standard operations.
To construct, \texttt{KademliaApp} requires the URI address of the external
application's interface; and the environment on which it works.

\begin{figure}[tbp]
\begin{lstlisting}
public synchronized void startUp() throws IOException {
  List<String> runCommand = new LinkedList<>();
  runCommand.add(mJavaCommand);
  runCommand.add("-Dorg.slf4j.simpleLogger.logFile=" + LOG_FILE);
  runCommand.add("-Dorg.slf4j.simpleLogger.defaultLogLevel=trace");
  runCommand.add("-cp");
  runCommand.add("lib/*:Kademlia.jar");
  runCommand.add("me.gregorias.Kademlia.interfaces.Main");
  runCommand.add("Kademlia.xml");
  mProcess = mKademliaEnv.runCommandAsynchronously(runCommand);
}

public Collection<NodeInfo> findNodes(Key key) throws IOException {
  Client client = ClientBuilder.newClient();
  WebTarget target = client.target(mUri).path("find_nodes/" + key.toInt());

  NodeInfoCollectionBean beanColl;
  try {
    beanColl = target.request(MediaType.APPLICATION_JSON_TYPE)
        .get(NodeInfoCollectionBean.class);
  } catch (ProcessingException e) {
    throw new IOException("Could not find node.", e);
  }
  NodeInfoBean[] beans = beanColl.getNodeInfo();
  Collection<NodeInfo> infos = new LinkedList<>();
  for (NodeInfoBean bean : beans) {
    infos.add(bean.toNodeInfo());
  }
  return infos;
}
\end{lstlisting}
\caption{\texttt{KademliaApp}: start a remote Ghoul instance; and use
HTTP-RPC to find neighboring nodes of the instance.}
\label{fig:app_example}
\end{figure}

\paragraph{\texttt{KademliaEnvironmentPreparator}} (Figure~\ref{fig:prepare})
\texttt{KademliaEnvironmentPreparator} copies Kademlia's dependency jar files and configuration
files to to the target environment.

\texttt{KademliaEnvironmentPreparator} uses methods of \texttt{Environment} to deploy the application.
graph;
\begin{figure}[tbp]
\begin{lstlisting}
public void prepare(Collection<Environment> envs)
    throws IOException {
  Collection<Environment> preparedEnvs = new LinkedList<>();
  Environment firstEnv = findFirstEnvironment(envs);
  for (Environment env : envs) {
    XMLConfiguration xmlConfig = prepareXMLConfiguration(env, firstEnv);
    try {
      xmlConfig.save(XML_CONFIG_FILENAME);
      Path localConfigPath = FileSystems.getDefault().getPath(LOCAL_CONFIG_PATH).toAbsolutePath();
      env.copyFilesFromLocalDisk(localConfigPath, ".");
      env.copyFilesFromLocalDisk(LOCAL_JAR_PATH.toAbsolutePath(), ".");
      env.copyFilesFromLocalDisk(LOCAL_LIBS_PATH.toAbsolutePath(), ".");
      preparedEnvs.add(env);
    } catch (ConfigurationException | IOException e) {
      clean(preparedEnvs);
      throw new IOException(e);
    }
  }
}
\end{lstlisting}
\caption{\texttt{KademliaEnvironmentPreparator} (fragment): prepare remote environments by copying jars and configuration files. Methods \texttt{restore}, \texttt{collectOutput}, \texttt{cleanOutput} and \texttt{cleanAll} are similar.}
\label{fig:prepare}
\end{figure}


\paragraph{Main and \texttt{KademliaAppFactory}} To glue those we also need to
define a simple factory which instantiates \texttt{KademliaApp}s given prepared
\texttt{Environments}. Also the entry point to dfuntest application - the main
method - needs to be defined.

\subsection{Test script}

We show how to check whether the topology graph induced by Kademlia routing
tables is connected (consistent).

To define the testing scenario, the tester writes the \texttt{run} method defined by the
\texttt{TestScript} interface (Figure~\ref{fig:core}).

\begin{figure}[tbp]
\begin{lstlisting}
public TestResult run(Collection<KademliaApp> apps) {
  try {
    startUpApps(apps); // Call startUp method of each KademliaApp.
    Thread.sleep(START_UP_DELAY_UNIT.toMillis(START_UP_DELAY));
  } catch (IOException e) {
    return new TestResult(Type.FAILURE,
        "Could not start up applications.", e);
  }
  try {
    startKademlias(apps); // Send start HTTP-RPC to each interface
  } catch (KademliaException | IOException e) {
    shutDownApps(apps);
    return new TestResult(Type.FAILURE, "Could not start Kademlias.", e);
  }
  mResult = new TestResult(Type.SUCCESS,
      "Connection graph was consistent the entire time.");
  scheduleCheckerToRunPeriodically(new ConsistencyChecker(apps))
  waitTillCheckerFinished();
  stopKademlias(apps);
  shutDownApps(apps);
  return mResult;
}

private class ConsistencyChecker implements Runnable {
  @Override
  public void run() {
    Map<Key, Collection<Key>> graph;
    try {
      // Call GET_ROUTING_TABLE RPC of each node
      graph = getConnectionGraph(mApps);
      ConsistencyResult result = checkConsistency(graph);
      if (result.getType() == ConsistencyResult.Type.INCONSISTENT) {
        mResult = new TestResult(Type.FAILURE,
            "Graph is not consistent starting from: " + result.getStartVert()
            + " could only reach " + result.getReachableVerts().size());
        shutDown();
        return;
      }
    } catch (IOException e) {
      mResult = new TestResult(Type.FAILURE,
          "Could not get connection graph: " + e + ".");
      shutDown();
      return;
    }
    --mCheckCount;
    if (mCheckCount == 0) {
      shutDown();
    }
  }
  ...
}
\end{lstlisting}
\caption{\texttt{KadmliaConsistencyTestScript} (fragment) periodically checks consistency of the induced Kademlia graph.}
\label{fig:core}
\end{figure}

\subsection{Dfuntest validation tests}

We configured Ghoul build system to create a separate .jar package with the above dfuntest.
The tests are executed as a standard Java application.
The main method expects an XML configuration file as its argument.
This configuration file contains all pertinent parameters for setting up
environments and controlling test execution.
For example to start a test suite on 7 remote hosts we provide an XML
configuration file with following information:

\begin{verbatim}
<SSHEnvironmentFactory>
  <hosts>
    roti.mimuw.edu.pl,
    prata.mimuw.edu.pl,
    planetlab2.wiwi.hu-berlin.de,
    planetlab2.s3.kth.se,
    planetlab1.u-strasbg.fr,
    planetlab01.tkn.tu-berlin.de,
    pl1.uni-rostock.de
  </hosts>
  <username>
    mimuw_user
  </username>
  <privateKeyPath>
    /home/mimuw_user/.ssh/id_rsa
  </privateKeyPath>
</SSHEnvironmentFactory>
\end{verbatim}

The rest is handled by the dfuntest application and at the end a human-readable
report directory is produced. This report directory contains a separate
directory for each executed TestScript test with logs and summary report.
Additionally there's a summary for all the executed tests
(Figure~\ref{fig:sumrep}).

\begin{figure}[tbp]
\begin{verbatim}
[SUCCESS] KademliaChurnConsistencyCheckTestScript
[SUCCESS] KademliaConsistencyCheckTestScript
[SUCCESS] KademliaDataReplicationTestScript
\end{verbatim}
\caption{A summary report for the test in which all \texttt{TestScript}s passed}
\label{fig:sumrep}
\end{figure}

We have already explained \texttt{KademliaConsistencyCheckTestScript}. 
A variation of that test is:
\texttt{KademliaChurnConsistencyCheckTestScript}. 
The churn version additionally simulates churn in-between consistency checks by
restarting nodes with 50\% probability.\\
\texttt{KademliaDataReplicationTestScript} makes each node put a key-value pair
into distributed storage and checks whether on \texttt{GET} request the data is
replicated.

Because a successful result is uninteresting we'll show what happens on failure.
At the same time we verify that our tests actually test and not just display
success.
To inject a failure into Ghoul, we used a too small bucket size (1, compared
with a usual value of 5).
Such a small bucket should make the graph inconsistent, because node finding
messages will have too little diversity.
This set up led to failure of all tests (occassionally data replication works
fine in that scenario).
The summary for all tests (\ref{fig:sumrep_failure}) shows failure in all
tests.
The summary for consistency check (Figure~\ref{fig:conrep}) clearly points to a
misbehaving node (19). 
This points the tester where to look to the source of this error.

\begin{figure}[tbp]
\begin{verbatim}
[FAILURE] KademliaChurnConsistencyCheckTestScript
[FAILURE] KademliaConsistencyCheckTestScript
[FAILURE] KademliaDataReplicationTestScript
\end{verbatim}
\caption{A summary report for the test in which all \texttt{TestScript}s have
failed}
\label{fig:sumrep_failure}
\end{figure}

\begin{figure}[tbp]
\begin{verbatim}
[FAILURE] Found inconsistent graph.
The connection graph was:
18: [16, 20, 24, 0]
19: [16, 20, 24, 0]
16: [17, 18, 20, 24, 0]
17: [16, 18, 20, 24, 0]
22: [23, 20, 16, 24, 0]
23: [22, 20, 16, 24, 0]
20: [21, 23, 16, 24, 0]
21: [20, 23, 16, 24, 0]
24: [16, 0]
2: [3, 0, 4, 11, 16]
3: [2, 0, 4, 9, 16]
0: [1, 2, 4, 8, 16]
1: [0, 2, 4, 8, 16]
6: [7, 4, 0, 11, 16]
7: [6, 4, 0, 9, 16]
4: [5, 6, 0, 9, 16]
5: [4, 6, 0, 8, 16]
10: [8, 12, 0, 16]
11: [8, 12, 0, 16]
8: [9, 10, 12, 0, 16]
9: [8, 10, 12, 0, 16]
14: [15, 12, 8, 0, 16]
15: [14, 12, 8, 0, 16]
12: [13, 14, 8, 0, 16]
13: [12, 14, 8, 0, 16]

It's strongly connected components are:
[19]
[18, 16, 17, 22, 23, 20, 21, 24, 2, 3, 0, 1, 6, 7, 4, 5, 10, 11, 8, 9, 14, 15, 12, 13]
\end{verbatim}
\caption{A report generated after KademliaConsistencyCheckTestScript fails.}
\label{fig:conrep}
\end{figure}

In case an error requires a debugging process the developer may easily change
the testing environment and test parameters by changing the configuration file
to facilitate the bug triage.

\section{Performance evaluation}
\label{sec:performance}

\subsection{Registration speed}
\begin{figure}[tbp]
  \centering
\resizebox{\columnwidth}{!}{\input{regtest.pdf_tex}}
\caption{TODO}
\label{fig:reg_test}.
\end{figure}
\subsection{Find node speed}

\chapter{Conclusion and future work}
\label{ch:conclusion}
We have presented Ghoul - a security extension of Kademlia distributed hash table protocol.
We also provide a prototype implementation of Ghoul, i.e. message extensions, certificate generation and management, and registrars.
This prototype allows for deployment of Kademlia DHT with centralized gatekeeper, where node's entry is may be controlled and key-generation is guaranteed to be random.

The main goal of the prototype is to show that it is possible to achieve secure DHT without requiring additional work from the user.
We believe that the example shown in chapter \ref{ch:implementation} proves that, indeed, running Ghoul is as simple as running a bare Kademlia node.

In the future we plan to finish the implementation with SybilControl protocol and release the library publicly.

Additionally, we have presented the dfuntest's design pattern and implementation for writing distributed tests with centralized control.
Dfuntest offers a coherent and expressive abstraction for distributed testing.
This abstraction allows for clean automation of the testing process which in turn also gives greater control of the real-world testing environment.

The main goal of dfuntest are jUnit-like acceptance tests: set up environments,
run the tested application, check some assertions on the state. However, thanks
to flexible test scripts, we are currently using dfuntest also for performance
evaluation of various subsystems of
nebulostore\footnote{\url{http://nebulostore.org}}, which previously required ad-hoc deployment scripts.

In the immediate future we plan to extend dfuntest library with ability to run tests of heterogeneous applications in heterogeneous environmnents---e.g., a client-server application where capabilities of environments are different.
We also plan better mechanisms for coping with failing environments: a failure of remote host, independent from the application. Currently such failure can only be detected with ssh queries.

\begin{thebibliography}{99}
\addcontentsline{toc}{chapter}{Bibliography}

\bibitem[Awe10]{awe10}
  Baruch Awerbuch, Christian Scheideler,
  \textit{Robust Random Number Generation for Peer-to-Peer Systems},
  Distributed Computing and Networking vol. 5935, 2010, p.195-206

\bibitem[Baz05]{baz05}
  Rida A. Bazzi, Goran Konjevod,
  \textit{On the establishment of distinct identities in overlay networks},
  Proceedings of the 24th Symposium on Principles of Distributed Computing,
  2005

\bibitem[Cas02]{cas02} Castro, Druschel, Ganesh, Rowstron, Wallach,
  \textit{Secure routing for structured peer-to-peer overlay networks},
  ACM SIGOPS Operating Systems Review - OSDI '02: Proceedings of the 5th
  symposium on Operating systems design and implementation,
  Vol. 36 Issue SI, Winter 2002,
  299-314

\bibitem[Con06]{con06}
  Tyson Condie, Varun Kacholia, Sriram Sankararaman, Joseph M. Hellerstein,
  Petros Maniatis,
  \textit{Induced churn as shelter from routing table poisoning},
  Proceedings of the 13th Symposium on Network and Distributed System Security,
  2006
 
\bibitem[Din06]{din06}
  Jochen Dinger and Hannes Hartenstein,
  \textit{Defending the Sybil attack in P2P networks: Taxonomy, challenges, and
  a proposal for self-registration},
  Proceedings of the 1st International Conference on Availability, Reliability
  and Security,
  2006

\bibitem[Dou02]{dou02} John R. Doucer,
  \textit{The Sybil Attack},
  Proceedings of the 1st International Workshop on Peer-to-Peer Systems,
  Lecture Notes on Computer Science, vol.  2429. 251–260,
  2002

\bibitem[Gol08]{gol08}
  Shafi Goldwasser, Mihir Bellare,
  \textit{Lecture Notes on Cryptography},
  2008


\bibitem[Hai06]{hai06} 
Haifeng Yu, Michael Kaminsky, Phillip B. Gibbons, Abraham Flaxman,
\textit{Sybilguard: Defending against sybil attacks via social networks},
ACM SIGCOMM ’06, p. 267--278

\bibitem[Li12]{li12}
Frank Li, Prateek Mittal, Matthew Caesar, Nikita Borisov,
\textit{SybilControl: Practical Sybil Defense with Computational Puzzles},
arXiv:1201.2657, 2012

\bibitem[Mac09]{mac09}
  Leonardo Maccari, Matteo Rosi, Romano Fantacci, Luigi Chisci, Luca Maria
  Aiello, Marco Milanesio,
  \textit{Avoiding Eclipse attacks on Kad/Kademlia: an identity based approach},
   IEEE International Conference on Communications, 2009

\bibitem[May02]{may02}
  Petar Maymounkov and David Mazieres,
  \textit{Kademlia: A Peer-to-Peer Information System Based on the XOR Metric},
  IPTPS '01 Revised Papers from the First International Workshop on Peer-to-Peer
  Systems, p. 53-65

\bibitem[Nar01]{nar01}
  T. Narten and R. Draves,
  \textit{Privacy Extensions for Stateless Address Autoconfiguration in IPv6},
  \texttt{http://tools.ietf.org/html/rfc3041},
  2001

\bibitem[Row01]{row01}
  Antony Rowstron and Peter Druschel,
  \textit{Pastry: Scalable, decentralized object location, and routing for
  large-scale peer-to-peer systems}, 2001

\bibitem[Sin06]{sin06}
  Atul Singh, Tsuen-wan "johnny Ngan, Peter Druschel, Dan S. Wallach,
  \textit{Eclipse attacks on overlay networks: Threats and defenses},
  In Proceedings of the 25th Annual Joint Conference of the IEEE Computer and
  Communiations Societies, 2006

\bibitem[Sit02]{sit02}
  Emil Sit, Robert Morris,
  \textit{Security Consideration in Peer-to-Peer Distributed Hash Tables},
  Revised Papers from the First International Workshop on Peer-to-Peer Systems,
  IPTPS '01, 2002, p. 261--269

\bibitem[Sto03]{sto03}
  Ion Stoica; Robert Morris, David Karger, M. Frans Kaashoek, Hari Balakrishnan,
  \textit{Chord: A Scalable Peer-to-peer Lookup Service for Internet
  Applications},
  IEEE/ACM Transactions on Networking, Vol. 11, 2003

\bibitem[Tim11]{tim11}
  Juan Pablo Timpanaro, Thibault Cholez, Isabelle Chrisment, Olivier Festor,
  \textit{When KAD meets BitTorrent - Building a Stronger P2P Network},
  HotP2P 2011, May 2011, Anchorage, ALASKA, United States.

\bibitem[Urd11]{urd11} Guido Urdaneta, Guillaume Pierre, Maarten Van Steen
\textit{A Survey of DHT Security Technologies}, ACM Comput. Surv.  43, 2,
Article 8 (January 2011)

\bibitem[Wan05]{wan05}
  Honghao Wang, Yingwu Zhu, Yiming Hu,
  \textit{An efficient and secure peer-to-peer overlay network}, 
  Proceedings of the 30th Local Computer Networks,
  2005

\bibitem[Wan08]{wan08}
  Peng Wang, James Tyra, Eric Chan-Tin, Tyson Malchow, Denis Foo Kune, Nicholas
  Hopper, Yongdae Kim,
  \textit{Attacking the Kad Network},
  Proceedings of the 4th International Conference on Security and Privacy in
  Communication Networks, SecureComm '08, article no. 23

\bibitem[Wan12a]{wan12a}
  Liang Wang and Kangasharju, J.,
  \textit{Real-world sybil attacks in BitTorrent mainline DHT},
  Global Communications Conference (GLOBECOM), 2012 IEEE, p. 826-832

\bibitem[Wan12b]{wan12b}
   Qiyan Wang and Nikita Borisov
   \textit{Octopus: A Secure and Anonymous DHT Lookup}
   IEEE 32nd International Conference on Distributed Computing Systems (ICDCS), 2012

\bibitem[Wan13]{wan13}
  Liang Wang and Jussi Kangasharju,
  \textit{Measuring Large-Scale Distributed Systems: Case of BitTorrent
  Mainline DHT}, 
  IEEE Peer-to-Peer, 2013


\end{thebibliography}




\end{document}

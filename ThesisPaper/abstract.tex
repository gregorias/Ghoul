\begin{abstract}
Topology creation and maintenance protocols are one of the first choices a
designer of a Peer-to-Peer system has to face. Distributed hash tables, such as
Chord or Kademlia, are frequently chosen due to their good performance.
Unfortunately these core protocols rely on correct behavior of their
participants and are prone to malicious behavior. A completely unprotected
system may be controlled or shutdown by even a modest adversary.

There exists a rich set of proposals which aim to secure DHTs, and yet most of
the popular and widely used Peer-to-Peer applications using DHTs, such as:
Bitorrent Mainline DHT, Kad, OpenDHT, are not using them and are known to be
insecure. I feel that this is the case, because these solutions often are found
to be impractical by developers. Impracticalities include things such as:
suboptimal performance, too stringent assumptions, i.e. reliance on central
authority, but most important of all is lack of good implementation.

This goal of this work is to design and implement an extensible DHT system
that is protected against wide array of attacks, doesn't need to rely on
central, trusted host, and can self-recover in case an attack happens.  This
implementation is done using Java using tested software engineering practices
that provide elasticity and customizability necessary to allow easy inclusion of
the library into an application.

[TODO rezultaty (extensibility?)]
\end{abstract}

\chapter{Ghoul system description}
\section{Goals}
I decided to build Ghoul to fill a niche of secure open-source DHT network
implementation. In my experience it is difficult to find a good open-source
implementation of Kademlia. By good I mean one that satisfies all of the
following criterias:

\begin{description}
  \item{\textbf{Simplicty}} Implementation of Kademlia should do only one thing:
    operate according to the Kademlia protocol.
  \item{\textbf{Extensibility}} If I want to extend the network protocol with a
    NAT traversal mechanism I should not be required to modify large portion of
    the library network communication stack.
  \item{\textbf{Documentation}} An open-source project should have maintained
    documentation to help others contribute or use it.
  \item{\textbf{Robustness}} Implementation should not break after it is used
    for 1 hour on small sized network in Planetlab.
\end{description}

I couldn't find an implementation that satisfies all of the above in Java and
none that would satisfiably claim to be secure in any language.

Solutions used in KAD network and designed for Mainline DHT use IP address to
prevent Sybil attack and arbitrary ID choice, this works against unsophisticated
attacker, but has serious problems stemming from the fact that IP was not
designed for that use case. IP spoofing, IPv6 privacy extensions and lack of
anonymity are serious concerns that either disarm such solution or introduce a
serious flaw. I feel that lack of adoption of security measures that are present
in the research literature stems from the stereotype that such measures are
either hard to implement, slow, hard to satisfy their dependencies, or
needlessly complicate use of the application. With Ghoul I show that it is
possible to design and build a DHT that satisfies following constraints:

\begin{description}
  \item{\textbf{Security}} The DHT should be resistant to common Sybil and
    eclipse attacks.
  \item{\textbf{Performance}} The DHT should be as performant as the original
    protocol where possible, especially in common use cases of finding a node
    and data storage. In other cases the time of operation should be practical,
    meaning that time or message complexity should not be greater than $O(\log
    n)$ for network of size $n$.
  \item{\textbf{Simplicity}} It should be simple to use. User should not be
    forced to perform an extensive set up, or prepare a large configuration file
    to just start the DHT.  The entire scheme should be simple to describe. User
    or developer should be able to understand how the protection schemes work
    and be able to reason how it would behave in his application.
  \item{\textbf{Extensibility}} The application develop should be able to extend the
    system with his own security mechanism easily if so desired.
  \item{\textbf{Anonymity}} The security mechanism should not sacrifice user
    anonymity.
\end{description}

I assume standard threats and network conditions that is:

\begin{itemize}
  \item An attacker may introduce colluding nodes into the system, but has
    limited computing and communication power.
  \item For network communication there is point-to-point message passing
    primitive which may fail to deliver and does not give any bound on delivery
    time.
  \item Nodes may fail arbitrarily.
\end{itemize}

In this chapter I present protocols that create Ghoul.

\section{Overview}
Ghoul system is built on top of Kademlia DHT protocol. Each node in the system
is an independent agent which acts as a Ghoul node and communicates with other
nodes using messages sent over network transmission.

There two kinds of nodes in the system: a DHT node and a registration authority.
The DHT node is the Kademlia node extended with security measures. It initiates
and maintain its routing table and answers queries. Additionally each node is
responsible for maintaining the security of the entire network by participation
in P2P certificate revocation propagation, maintenance of its security
certificate, and answer in security queries.

A small set of nodes, called registration nodes, is responsible for facilitating
node’s join into network by using distributed random key generation and
certificate issuance. Registration nodes are distributed version of central
authority, with a twist that they a few of them might be malicious. Their
existence is motivated by the fact that it is very hard to guarantee random key
generation and verification in distributed setting, on the other hand I want to
provide administrative scalability where application's owner may allow other
parties to provide their registration nodes for the protocol and not worry
whether they have pure intentions as long as a single node is honest.

\section{Data structures}

\subsection{Public key cryptography}
Ghoul uses public key cryptography for node authentication and certification.
Every DHT node and registration node generates a public/private key pair on its
first entry. The private key is always kept private, but public can and should
be propagated. Additionally public keys of registration nodes are known by all
nodes.

To participate in the network a DHT node requires a \textbf{certificate}.
Certificate is a tuple of form:
(\texttt{NODE\_PUB\_KEY},~\texttt{NODE\_DHT\_KEY},~\texttt{EXPIRATION\_DATETIME})
signed with a registration node's private key such that it is known which node
signed it. Such certificate is given by the registration node together with
\texttt{NODE\_DHT\_KEY} as a proof that that node is valid and its key was
generated randomly by the registration node or in a distributed way by a set of
registration nodes. If the key is generated by a group of registration nodes
then all of them provide a certificate.

We say that node is valid if it has at least $m$ non-expired, non-revoked
certificates from different hosts. The parameter $m$ should be set to such a
value that we are sure that there might at most $m-1$ malicious registration
nodes.

Every node maintains a store for valid certificates that it has received. This
store may be limited in size. In such a case a node may remove certificates last
seen node's or certificates close to expiration. Additionally every node in the
routing table must have a valid certificate in the store, otherwise it should be
removed. If a neighbors certificate is close to expiration than a \texttt{PING}
should be sent to get an unpdated certificate.

\subsection{Messages}
Ghoul extends each DHT message with following fields:
\begin{description}
  \item{\texttt{certificateRequest}} Signifies that a response to this message
    should contain a list of certificates. This field is set when the recipient
    is considered to be a neighbor or potential neighbor and either its
    certificates are not present, not valid, or close to expiration.
  \item{\texttt{certificates}} A list of valid certificates provided by the
    registration authority.
  \item{\texttt{signature}} Signature signed by the sender's private key.
\end{description}

Just like normal Kademlia messages are used for routing table maintenance so
similarly in Ghoul we attach security information to them. Only the list of
certificates may have a larger size (each certificate has a size of $\approx$
1KB), but it is sent only sporadically.

Each one of those additional fields is optional. If the signature field is not
provided it means that the node is not interested in being part of the network
and instead just uses it to find information. That is considered normal
behavior, such node is not added to the routing table (it may have possibly not
even provided its key information), and is simply responded with appropriate
response. Otherwise every node should always sign its messages.

The \texttt{certificateRequest} field is set if all of the following conditions
are met:

\begin{itemize}
  \item The recipient's certificate is not present in the store or is close to
    expiration.
  \item The recipient is in the routing table or may be considered a candidate.
\end{itemize}

If the \texttt{certificateRequest} field is set then the recipient should attach
list of certificates to its response or if no response would be normally given
then it should respond with a \text{PONG} message.

If a response to the message with the \texttt{certificateRequest} does not have
a valid list of certificates then it is discarded.

\section{Protocols}
\subsection{Challenge protocol}
In some neuralgical and vulnerable points in the protocol we may intent to limit
the number of requests a node can make using cryptographic challenges. This
achieves 2 goals: It significantly increases of a cost of an attack which relies
on number of queries, such as a Sybil attack, and provides a throttle mechanism
to prevent a node overload.

Whenever the protocol requires the node sending a query or a request to provide
a proof-of-work for this request the challenge protocol is used
(\ref{fig:chal_prot}).

\begin{figure}
\begin{msc}{Challenge protocol}
\setlength{\instdist}{9cm}
\setlength{\envinstdist}{3cm}
\declinst{client}{C}{Client}
\declinst{server}{S}{Server}
\mess{Send challenge request}{client}{server}
\nextlevel
\action*{
\begin{minipage}{4cm}\centering
Choose a random nonce\\
$r\in\{0,1\}^*$, difficulty $p \in \N$, $t := \text{current time}$.
\end{minipage}}{server}
\nextlevel[11]
\mess{Send challenge $m_2 := (t, r, p)_{s_{pr}}$}{server}{client}
\nextlevel[1]
\action*{
\begin{minipage}{4cm}\centering
  Find $s$ such that $H(r || s)$ has $p$ last bits equal to zero.
\end{minipage}}{client}
\nextlevel[4]
\mess{Send solution $m_3 := (m_2, s, c_{pub})_{c_{pr}}$}{client}{server}
\nextlevel
\action*{
\begin{minipage}{4cm}\centering
Check the validity of the solution. Save the nonce into active puzzles list
until the validity of the puzzle has ended.
\end{minipage}}{server}
\nextlevel[4]
\end{msc}
  \caption{Challenge protocol}
  \label{fig:chal_prot}
\end{figure}

The validity check consists of: verifying signature of $m_2$, checking whether
nonce is in not in the active puzzles list, checking time, verifying the
solution.

$m_3$ serves as a proof-of-work and may be accompanied by an additional request
to the server.

Note that there is a variant to this scheme in which we save distributed nonces
and later check that they exist in the list during verification. Depending on
the duration of the timeout and number of possible queries per second this may
provide a denial-of-service vulnerability.

\subsection{Robust distributed random key generation}

In Ghoul honesty of registrars is not assumed. In case of a single centralized
registrar such a node would greatly facilitate in an eclipse attack by
assigning to colluding nodes keys close to a target. Therefore a verifiable
distributed scheme for key generation is needed.

Here  I will present a distributed protocol which generates a random key that is
unbiased if at least one node is honest. More formally:

\begin{theorem*}
  Given a list $R = \langle r_1, \ldots, r_n \rangle$ of $n \geq 2$ nodes known
  to all nodes in $R$ the random key generation protocol generates an unbiased
  random key known to all participants if at least one node is honest and all
  nodes respond according to the protocol in a timely manner. Even in the case
  of a byzantine failure then all nodes that generate a key generate the same
  key that is unbiased if at least one non-failing node is honest.

  The chance that the malicious participant may manipulate the protocol so that
  this theorem does not hold is cryptographically negligible.
\end{theorem*}

The protocol that satisfies those properties is defined in
\ref{fig:key_gen_alg}.  It uses bit-commitment scheme which is defined as
follows:

\begin{defin*}[Bit-commitment]
  Bit-commitment is a cryptographic protocol that allows one to commit a value
  such that it is hidden and may be shared with others. Later the commiter may
  reveal the commited value to participants in such a way that they can check
  that it is indeed hidden in the earlier shared commitment. If the commiter
  tries to reveal a different value then this fact is detectable.

  Typical bit-commitment use is split into two phases:
  \begin{description}
    \item{\textbf{Hiding phase}} Commiter chooses a value and creates a
      commitment which hides the value. The commiter sends the commitment to
      interested parties.
    \item{\textbf{Reveal phase}} The value and the key used to create the
      commitment are revealed. Interested parties may check that they are
      consistent with earlier commitments.
  \end{description}

  Bit-commitment protocols may be either unconditionally binding, meaning even
  computationally unbounded commiter may not reveal a different value than the
  one commited, or unconditionally hiding, meaning no participant may find out
  the commited value. It can't be both.
\end{defin*}

The random number generation protocol consists of 3 phases:
\begin{enumerate}
  \item Commit and broadcast random number used to generate the key. This phase
    is used so that every node commits to its random number and can not change
    it later.
  \item Broadcast all received messages to others. This step ensures that every
    honest node has received the same set of messages and will generate the same
    key. If any node has sent inconsistent messages to honest nodes
    than this will be detected here.
  \item Broadcast key to unlock commitment.
\end{enumerate}

One may recognize that this protocol consist of the first two phases of a
3-phase byzantine failure protection protocol. Which is the reason why it has
its non-malleability properties.

\begin{figure}
  \begin{algorithmic}[1]
  \STATE $k_i \leftarrow $ random key
  \STATE $(h_i, s_i) \leftarrow $ bit commitment string and solution commiting
  $k_i$
  \FOR{$j := 1$ \TO $n$, $j \neq i$}
    \STATE Send $(h_i)_{i}$ to $r_j$
  \ENDFOR
  \STATE Wait $T$ seconds till all messages of the form $(h_j)_j$ are received
  from other nodes. If timeout happens then abort.
  \FOR{$j := 1$ \TO $n$, $j \neq i$}
    \STATE Send $\left( (h_1)_1, \ldots, (h_n)_{n}, k_i, s_i\right)_i$ to $r_j$
  \ENDFOR
  \STATE Wait $T$ seconds till all messages of the above form are received
  from other nodes. If timeout happens then abort. If any message is not
  coherent to the rest then abort.
  \FOR{$j := 1$ \TO $n$, $j \neq i$}
    \IF{$b_j$ is not a bit commitment of $k_j$}
      \STATE Abort protocol.
    \ENDIF
  \ENDFOR
  \STATE $k \leftarrow k_1 \oplus \ldots \oplus k_n$
\end{algorithmic}
  \caption{Distributed key generation algorithm generating random key $k$}
  \label{fig:key_gen_alg}
\end{figure}

An example of this protocol for 3 registrars is shown in
\ref{fig:key_gen_example}.

\begin{figure}
\begin{msc}{Distributed key-generation}
\setlength{\instdist}{5.5cm}
\setlength{\envinstdist}{3cm}
\declinst{ra}{}{RA}
\declinst{rb}{}{RB}
\declinst{rc}{}{RC}
\action*{
\begin{minipage}{4cm}\centering
  Generate random key $r_A$ and generate bit-commitment string for that key:
  $h_A$.
\end{minipage}}{ra}
\nextlevel[5]
\mess{}{ra}{rb}
\mess{$(h_A)_A$}{ra}{rc}
\nextlevel[2]
\mess{$(h_B)_B$}{rb}{ra}
\nextlevel[2]
\mess{$(h_C)_C$}{rc}{ra}
\nextlevel[2]
\mess{}{ra}{rb}
\mess{$((h_A)_A, (h_B)_B, (h_C)_C)_A$}{ra}{rc}
\nextlevel[2]
\mess{$((h_A)_A, (h_B)_B, (h_C)_C)_B$}{rb}{ra}
\nextlevel[2]
\mess{$((h_A)_A, (h_B)_B, (h_C)_C)_C$}{rc}{ra}
\nextlevel[2]
\mess{}{ra}{rb}
\mess{$(r_A)_A$}{ra}{rc}
\nextlevel[2]
\mess{$(r_B)_B$}{rb}{ra}
\nextlevel[2]
\mess{$(r_C)_C$}{rc}{ra}
\nextlevel[2]
\action*{
\begin{minipage}{4cm}\centering
  If messages are consistent generate key $r := r_A \oplus r_B \oplus r_C$.
\end{minipage}}{ra}
\nextlevel[4]
\end{msc}
\caption{Example of distributed key-generation for 3 registrars}
\label{fig:key_gen_example}
\end{figure}

The protocol uses $O\left(n^2\right)$ messages and is vulnerable to an attack in
which an attacker simply does not participate in it. However in the large scope
of the system it's not an issue, because we assume that introducing a registrar
requires an organizational approval which is a nontrivial barrier of entry and
later any malicious nodes may be easily detected and removed from the
centralized lists should it become compromised.

\subsection{Node join protocol}

Every node in order to participate in the DHT network is required to have a
collection of node certificates from registrars. Certificate serves as a proof
to other nodes that the key in the node ID has been generated by that
registration authority and that the registration authority allowed this node to
join the DHT.

The system's threat model allows some registration authorities to be malicious
and potentially generate biased keys. The node join protocol prevents such nodes
from performing a successful attack. The node join protocol takes a constant $m$
iniicating maximal number of malicious registrars. The node join protocol then
requires that at least $m+1$ certificates are needed for the node to be
considered valid.

The node join protocol is used 

In figure~\ref{fig:node_join_prot} the node join protocol is shown for $m=1$.

\begin{figure}
\begin{msc}{Node join protocol}
\setlength{\instdist}{5.5cm}
\setlength{\envinstdist}{3cm}
\declinst{client}{C}{Client}
\declinst{ra}{RA}{Registrar A}
\declinst{rb}{RB}{Registrar B}
\mess{Perform challenge protocol}{client}{ra}
\nextlevel[2]
\mess{Perform challenge protocol}{client}{rb}
\nextlevel
\action*{
\begin{minipage}{4cm}\centering
  Has received nonces $r_A, r_B$ during the challenge protocol.
\end{minipage}}{client}
\nextlevel[4]
\mess{$(r_A, r_B)$}{client}{ra}
\nextlevel[2]
\mess{$(r_A, r_B)$}{client}{rb}
\nextlevel[2]
\mess{$(r_A, r_B)_A$}{ra}{client}
\nextlevel[2]
\mess{$(r_A, r_B)_B$}{rb}{client}
\nextlevel[2]
\mess{$(r_A, r_B)_B$}{client}{ra}
\nextlevel[2]
\mess{$(r_A, r_B)_A$}{client}{rb}
\nextlevel[2]
\referencestart{r}{robust key-generation protocol}{ra}{rb}
\nextlevel[2]
\gate[r][b]{out}{rleft}
\mess{\parbox{4cm}{Node certificates \\ signed by A and B\\}}{rleft}{client}
\nextlevel[1]
\referenceend{r}
\end{msc}
\caption{Node's join protocol}
\label{fig:node_join_prot}
\end{figure}

[TODO certificate refresh]

\subsection{Certificate revocation system}

\subsection{SybilControl - Limiting number of Sybil nodes with computational
puzzles}
\subsubsection{Goals}
  Verifiable random key generation in Kademlia is a defense against eclipse
  style attacks, which in Kademlia are much harder to launch than in other
  DHTs, such as Chord, due to nonstrict routing tables and redundant, iterative
  routing algorithm. [TODO citation]. However eclipse attack might not be
  necessary if attacker can perform a Sybil attack. Although this the scope of
  this attack will be limited by the centralized authority thanks to the use of
  computation puzzles, it just limits the rate of entry.

  The centralized authority allows for implementation of various anti-Sybil
  check schemes, but it increases responsibilities of those nodes which we might
  want to keep as simple as possible. So in order to provide a defense against
  Sybil attack Ghoul incorporates SybilControl into its mechanisms.

\subsubsection{SybilControl mechanism}
  SybilControl is fully described in \cite{li12} here I will describe the core
  mechanism as it is used in Ghoul. 

  To limit the number of Sybil nodes SybilControl requires that the nodes
  generate a cryptographic puzzle periodically and solve it. To verify that this
  puzzle is generated honestly it includes similarly generated puzzles propagated
  from its neighbors. The puzzle string for node $A$ is $C_{A-new}$, who has
  received puzzles $C_{B_i}$ from its neighbors $B_{i}$ where:

  \begin{eqnarray*}
      R_{A-new} &=& B_1||C_{B_1}||\ldots||B_n||C_{B_n}||r_A||C_{A-old}\\
    C_{A-new} &=& H\left(R_{A-new}\right)
  \end{eqnarray*}

  $r_A$ is a random string generated by $A$. $H$ is a one-way function such as
  \texttt{SHA-2}. This is to ensure that $r_A$ is not specially prepared to make
  the puzzle solving easy.

  The solution to the puzzle is such a string $S$ that:

  \[ h = H\left(A||C_A||S\right)\]

  has at least $p$ last bits equal to zero.

  The entire puzzle state $P$ consists of: the solution $S_P$, the challenge
  $C_P$ and the record state $R_P$. Generated puzzle states are kept for some
  globally defined for verification purposes.

  \paragraph{Neighbor node verification}
  To verify that a neighbor node has done its proof-of-work it is asked to send
  its latest puzzle state $P$. Then the asking node checks whether this is a
  valid, solved puzzle state and 

  \paragraph{Non-neighbor node verification}
  To verify that a non-neighbor node has solved its puzzle in the recent past we
  need to check that our recent challenge has influenced its puzzle. To do that
  we find a path between us and the non-verified node. We query nodes in the
  path for puzzle state history. The node verification is then equivalent to
  finding out in these puzzle state history all puzzles that have influenced the
  puzzle of the node under verification. The node is verified iff a puzzle that
  we have produced is among them.

\subsubsection{SybilControl in Ghoul}
  SybilControl requires additional consideration as to how it integrates with
  Kademlia.

  First off the protocol loosely defines what constitutes a neighbor. An
  uncareful implementation of SybilControl in which every cryptographic puzzle
  propagation is included in the generation of the next puzzle may increase the
  size of the puzzle size considerably. This kind of behavior may be used to
  slow down the DHT by gratuitous send of challenge updates to all nodes in the
  DHT. Li et al. do not take this into consideration, but this can be prevented
  by including only challenges received from nodes in the routing table and
  additionally limiting the number of included challenges to those that have
  been received in the last $2p$ seconds from the $i$ closest nodes. Where $p$
  and $i$ are globally defined constants.

  Secondly the protocol described in SybilControl has only local effects, that
  is a node that does not have an up to date puzzle will just not be included
  into the routing table. Since Ghoul uses a certification system with
  revocation it is possible for nodes to complain about others to registrars. If
  registrars find that given node is malicious or Sybil they may globally revoke
  its certificate, significantly increasing the cost of a Sybil attack.

  [About additional advantages of Sybil control in Ghoul]
  [Problem with revocation spam]

\chapter{DHT, its weaknesses, and defense mechanisms}
\section{Chapter overview}
  In this chapter I define what is a Distributed Hash Table and specify Kademlia
  DHT protocol, describe reasons and methods of attacks on DHT, and present
  examples of different classes of protection mechanisms. Finally examples of
  effectiveness of described attacks on popular, real-world DHT deployments are
  provided. Those examples show that securing DHT is still an open problem in
  practice.

\section{DHT overview}

  Distributed hash table is a type of peer-to-peer system providing topology
  creation and maintenance protocols. They are typically designed for and
  tightly coupled with data storage protocols for key-value pairs, hence the
  hash table part in the name. Popular DHT designs include: Chord~\cite{sto03},
  Pastry \cite{row01}, Kademlia \cite{may02}.

  A structure of a typical DHT consists of a three main components. First is a
  keyspace, a collection of identifiers such as 160-bit bitstrings. Those keys
  are an abstract identification of a node and serve to partition the keyspace
  into ownership ranges over the network. Second is an overlay network - a
  topology induced by routing tables. Routing tables contain pairings between
  abstract keys and network addresses. Neighborhood tables are kept and
  maintained by each node. The final component of DHT specification are
  protocols for routing table maintenance and search of nodes that are not
  neighbors.

  This kind of structure has certain desirable properties. Firstly, a DHT is
  completely decentralized. Every peer plays the same role and the protocol does
  not require any form of central coordinator. Secondly, DHTs are scalable. In
  most protocols the size of a routing table is on the order of $O(\log n)$,
  where $n$ is size of the network. Message count complexity of node search
  protocol is logarithmic as well. Thirdly, DHTs are fault tolerant. In fact
  their protocols are designed to be resilient to even significant churn in the
  network. Churn is the rate of nodes' join/leave speed. To counteract that
  phenomena the routing table should be updated often enough.
 
  \subsection{Kademlia}
  Kademlia \cite{may02} is a popular choice among many peer-to-peer
  applications. It is well-known for its use in BitTorrent and eDonkey
  file-sharing applications. Wang and Kangasharju \cite{wan13} measure the size
  of Kademlia used in BitTorrent to peak at almost 25 million users.

  A node in Kademlia network is represented by an n-bit binary key (most popular
  choices are $n = 128$ or $n = 160$). Kademlia defines a distance in the
  keyspace as the xor function. That is distance between two keys $x, y$ is

  \[ d(x, y) = x \oplus y\]

  The routing table is a collection of $n$ buckets, where each bucket has
  maximum size $k$ -~a~\mbox{system-wide} constant. 
  $i$'th bucket holds active and recently seen nodes such that the first most
  significant bit that is different from routing table's owner is at position
  $i$. More formally: the $i$'th bucket of node's with key $x$ holds nodes with
  keys $y$ such that $2^i \leq d(x, y) < 2^{i+1}$.

  Kademlia node implements 4 remote procedure call (RPC) queries. In this
  context RPC is an inter-process communication protocol that defines a query
  message and an algorithm producing a response message for the query. These
  query messages are:

  \begin{description}
    \item{\texttt{FIND\_NODE}} This query is parametrized by a key that the sender
      wants to find. The response should consists of up to $k$ keys closest to
      the searched key in queried node's routing table including the recipient
      node itself.
    \item{\texttt{PING}} This query serves as a heart-beat and node's liveness
      check. It should be answered with a \texttt{PONG} message.
    \item{\texttt{PUT\_KEY}} Orders the recipient to store given (key, value) pair
      in its local storage. This message should be sent to multiple nodes close
      to the pair's key so as to replicate the data.
    \item{\texttt{GET\_KEY}} Structurally the same as the \texttt{FIND\_NODE}
      message. The recipient should return $k$ closest neighbors plus a (key,
      value) pair if it's present in its storage.
  \end{description}

  The main protocol in Kademlia is the node search protocol shown in
  algorithm~\ref{alg:node_search_alg}. Its distinctive feature is that it is
  iterative and performs parallel queries. This form of natural redundancy is
  well suited for preventing attacks \cite{sit02}.

  [TODO no nie wiem czy ten algorytm teraz wygląda lepiej. Wydaje mi się mniej
    czytelny (przede wszystkim ze względu na długie zdania) a nie chcę schodzić
    na poziom ściśłego języka. Zapomniałem już co było nie tak w poprzednim
  opr{\'o}cz mniejszej ścisłości.]

  \begin{algorithm}
    \begin{algorithmic}[1]
    \STATE \texttt{candidates} $\leftarrow$ empty priority queue for candidate
    nodes ordered by closeness to key $y$
    \STATE Add all nodes from the routing table and this node to
    \texttt{candidates}
    \STATE Set added nodes' state to \texttt{unqueried}
    \WHILE {$|$nodes whose state isn't \texttt{responded} among top $k$
    candidates$| > 0$}
    \WHILE {$|$nodes with state \texttt{queried} among top $k$
      candidates$| < \alpha$ \AND \\
      $|$nodes with state \texttt{unqueried} among top $k$
            candidates$| > 0$}
        \STATE $n \leftarrow$ first \texttt{unqueried} node in
          \texttt{candidates} 
        \STATE Send \texttt{FIND\_NODE} to $n$
        \STATE $n$'s state $\leftarrow$ \texttt{queried}
    \ENDWHILE
    \COMMENT{maintain up to $\alpha$ parallel queries}
      
    \STATE $e \leftarrow$ response or timeout
    \IF{$e$ is a response}
      \STATE $n \leftarrow$ $e$'s sender
      \STATE $n$'s state $\leftarrow$ \texttt{responded}
      \STATE Add new candidates from response's neighbor list to
      \texttt{candidates}
    \ELSE
      \STATE $n \leftarrow$ timed-out node
      \STATE Remove $n$ from \texttt{candidates}
    \ENDIF
    \ENDWHILE
  \end{algorithmic}
    \caption{Node search protocol for key $y$}
    \label{alg:node_search_alg}
  \end{algorithm}

  The protocol for finding stored data associated with a key is structurally the
  same. Only, instead of \texttt{FIND\_NODE}, \texttt{GET\_KEY} is sent and the
  reply additionally contains stored data if any exists.

  Because the distance metric is symmetric Kademlia does not require specialized
  routing table maintenance messages. With every message receipt the recipient
  node adds the sender to its routing table if there is space in the appropriate
  bucket. If there is no space then it sends a \texttt{PING} message to the
  least recently seen node in the given bucket. If the intended recipient of the
  \texttt{PING} message times out then it is replaced by the candidate.

\subsection{Vulnerabilities}
  DHT's combination of Peer-to-Peer relationships and structural topology make
  them very vulnerable to attacks which abuse topological dependencies and
  structural constraints to facilitate them. The two main weaknesses of DHTs
  which prevention schemes try to rectify are:

  \begin{enumerate}
    \item Node identity is a virtual concept and lacks connection to the real
      world. DHTs as a distributed storage space rely on data, message
      redundancy, and cohesiveness of the topology. Both are easily broken if an
      attacker may either arbitrarily place itself in the topology and fill
      routing tables with colluding nodes (Eclipse attack) or get as many nodes
      as it wants (Sybil attack).

    \item Each node only maintains a local view over the network, which does not
      provide enough context to detect and prevent any malicious behavior. Often
      malicious behavior is indistinguishable from honest one unless we have
      knowledge about global state and history. Sometimes even that is not
      enough, for example verification of a specific key randomness is naturally
      unverifiable in a plain scheme.
  \end{enumerate}

\section{Eclipse attack}
  Attacks on DHT can have 3 different goals:
  \begin{description}
    \item{\textbf{denial of service}} The attacker aims to either stop the
      entire system from functioning or deny access to a part of it,
    \item{\textbf{spoofing}} The attacker wants to provide false data to other
      nodes or control them
    \item{\textbf{information}} The attacker wants to gather information about
      participants activity.
  \end{description}

  Those goals are greatly facilitated if an attacker controls the routing tables
  of other nodes. An attack which poisons routing tables of other nodes by
  disproportionately filling them with entries to colluding malicious nodes is
  called an eclipse attack. The name comes from one of its effects, if a node's
  routing table is filled with malicious nodes it is effectively partitioned
  from the rest of the network.

  \subsection{Prevention mechanisms}

  Sit and Morris \cite{sit02} have been among the first to discuss general
  guidelines against preventing attacks in DHT networks. They advise designing
  the system with verifiable constraints so that a node can distinguish
  malicious behavior from honest one and provide a proof to other nodes.
  Mechanisms, mainly data and node look up, should have a built-in redundancy so
  that a malicious node presence can be avoided in at least one path of
  execution. The lookup itself should be interactive and allow for clear view of
  progress, which allows a node to verify its correctness and initiate
  redundancy protocols if necessary. Finally, as in any distributed system,
  points of single responsibility should be avoided for obvious reasons.
  
  \subsubsection{Constrained routing table}
  Castro et al. \cite[p. 20]{urd11} have modified Pastry with additional
  constrained routing table and back up routing mechanism to protect against
  eclipse attacks. If node key assignment is truly random and routing table
  entries are narrowly specified then it is difficult for an attacker to inject
  malicious entries. DHTs such as Kademlia and Pastry are therefore more
  vulnerable than Chord in that regard.

  \subsubsection{Node scrambling}
  A different method to mitigate an eclipse attack is to constantly move node
  ids in unpredictable ways so that it is unfeasible to eclipse a local
  keyspace. Such solution is proposed by Condie et al. \cite[p. 21]{urd11}.
  Condie shows that a periodical random, unpredictable node reassignment
  augmented by techniques which mitigate negative churn effects, that is
  grouping nodes into different timeout groups so that timeouts are not
  synchronized and precalculation of future routing table, significantly
  improves Castro's solution.

  Awerbach and Scheideler \cite{awe10}, while proposing a robust, distributed
  random number generator, propose a scheme of global node rearrangement on each
  node join event. They show that using a robust cuckoo rule to move $O(\log n)$
  nodes makes the network resilient to join-leave attacks.

  \subsubsection{Topology graph analysis}
  Singh et al. \cite[p. 23]{urd11} observe that if malicious nodes form only a
  fraction of the network then during an Eclipse attack their in-degree must be
  higher than average in-degree of honest nodes. Therefore they propose a
  protocol which limits in-degree of nodes. Each node $x$ maintains an
  additional backpointer table, which contains nodes having $x$ in their routing
  tables. Every node $y$ periodically checks backpointer tables of their
  neighbors and if they are larger than a global constant or does not contain
  $y$ then it removes $x$ from its routing table.

  This scheme requires the backpointer check to be anonymous, therefore the
  Scheme uses a common technique of onion routing using anonymizer network to
  achieve that. Anonymizer are not assumed to be honest and therefore it
  possible that a honest node will be labeled as malicious with small
  probability.

  This scheme is effective at limiting eclipse attacks, but it requires the
  global in-degree bound to be tight, especially for large networks, which may
  decrease the efficiency of base protocol.

\section{Sybil attack}
  DHTs routing mechanisms rely on the assumption that nodes are placed uniformly
  inside its key space and only a fraction of the network may fail, either due
  to churn or node failure. In case of eclipse attack or routing attack
  prevention mechanism they additionally assume that only a given portion of the
  network may be malicious. Those assumptions can be easily violated in an open
  DHT.

  Firstly, key generation is not verifiable. There is no way for a node to
  verify whether given key has been generated randomly. It is possible to use a
  statistical test to verify uniformity constraints, but they are often not
  enough to accuse specific node of malicious behavior.

  Secondly, DHTs operate on an ephemeral sense of identity represented by a key.
  A Sybil attack happens when an attacker uses a single physical computer to
  represent multiple DHT nodes increasing his power without incurring additional
  cost. Since most protocols rely on honest nodes having a majority, a Sybil
  attack can shut down the system by even a modest attacker.

  Thirdly, DHTs agents use, like in any distributed system, only locally
  available information to determine their course of actions. However malicious
  behavior is often indistinguishable from correct one unless we have
  information about global state or history.

  Preventing a Sybil attack requires a gatekeeper or, more accurately, a
  bouncer. A protocol which prevents a single physical node to enter more than
  once and a way to throw out offenders. Achieving that in a fully distributed
  way is difficult. It is not possible to achieve adequate security if nodes
  only have access to their immediate surroundings. Such a limited view lacks
  necessary information to distinguish between benign and malicious behavior.
  Additionally it's much easier to simply eclipse a node. Protocols that
  overcome this obstacle by sharing information globally do so at the cost of
  performance and scalability. Such cost usually makes a DHT unpractical for
  larger networks. That's why most practical solutions make use of a central
  authority. This authority serves as a trusted entry point to the network, it
  generates signed random keys for newcomers and checks global constraints.

  Central authority has well-known drawbacks. Most of them stem from the fact
  that such authority represents a single point of failure. A failure of this
  authority may cause shut down of the network. Compromised authority is a
  security risk. It also acts negatively on scalability of the solution.
  
  Apart from technological disadvantages central authority is often an
  organizational problem. It forces an agent to maintain it in order to provide
  a service. This is done reluctantly since it makes the agent responsible for
  maintaining the service and Peer-to-Peer user community may be distrustful
  over centralized solution.

\subsection{Prevention mechanisms}

  The conclusion that only central authority solutions are practical for
  prevention of Sybil attacks was first deduced by \cite{dou02}, which also is
  the first paper studying this class of attacks \cite[p. 5]{urd11}

  \subsubsection{Centralized gatekeeper}
  Castro et. al. \cite{cas02} agree with \cite{dou02} conclusion and propose
  using a set of central trusted authorities which certify random key and public
  key pair. They suggest adding an IP address to the certificate, so that the
  certificate would be binded to physical host. Unfortunately this solution is
  known to be problematic. First of all IP address can be spoofed and with IPv6
  privacy extensions changed all together, rendering this solution completely
  insecure in the future. Second of all it would cause a problem for users
  behind NAT or with moving IP address. Third of all it hinders privacy
  protection mechanisms. Nevertheless using IP address is very easy to implement
  and practical for many current DHT networks.

  Centralized gatekeeper gives much flexibility in implementing an
  authentication and Sybil attack prevention scheme. For example there may be
  required a small monetary payment for entry into the DHT network or a
  computational puzzle may have to be solved.

  \subsubsection{Distinction based on network footprint}

  Without centralized authority network peers would have to register new nodes
  in a distributed way. To distinguish distinct identities from same ones they
  may use network characteristics. Such characteristics are usually easy to
  check in the sense that any node may do it, giving low barrier of entry to the
  network, but it carries an assumption that they are unfalsifiable, which is
  not always the case.

  A simple ID calculation based on IP address and port number is proposed by
  Dinger and Hartensteing \cite[p. 6]{dou02}. They propose that a node would
  register at $r$ Chord registration nodes whose position is calculated from
  hash of node's ID. If majority of registration nodes accept the new node then
  it is said to be accepted. A node registration fails if the number of nodes
  with given IP address exceeds a constant $a$. This solution requires all nodes
  to confirm validity of their neighbors.

  IP addresses may be spoofed and cause a problem for NAT users.
  Wang~et.~al.~\cite[p. 7]{dou02} proposed adding router's IP and MAC address
  and RTT measurements between the node a designated set of landmarks. A similar
  idea may be found in Bazzi~and~Konjevod~\cite[p. 7]{dou02} where nodes are put
  into $d$-dimensional space based on physical location of a node and a node are
  said to be distinct if their coordinates are far away enough.

  \subsubsection{Distinction based on social network}

  Another class of defenses uses a social network, a graph of trust
  relationships established by humans, to discover or mitigate a Sybil attack.
  In such defenses it is assumed that each node can verifiably identify itself
  as a node in the social network and that each edge forms a trust relationship
  which means that two people are friends and that they trust each other not to
  launch a Sybil attack. If any node was to launch a Sybil attack then the
  social network graph will exhibit special properties which may be detected. 

  Although such schemes show good performance, both in terms of application
  overhead and attack detectability, they are difficult to apply in real world
  applications. The biggest problem is to generate or acquire a social graph of
  sufficient quality. This also deprives users of anonymity which for many
  Peer-to-Peer application users is an important factor.

  SybilGuard \cite{hai06} mitigates effects of a Sybil attack by limiting the
  number of Sybil nodes and recognized Sybil groups. It defines the edges
  between honest nodes and Sybil nodes as attack edges and notes that in a
  social graph there can be only few such edges. Then it allows a node to
  partition its neighbors into classes and guarantees with high probability
  that a number of classes that contain a Sybil node is limited by the number of
  attack edges. Although the scheme does not detect Sybil nodes it gives a
  guarantee that with a sufficient redundancy an attacker will not be able to
  intercept a query or manipulate data.

  \subsubsection{Computational puzzles}
  SybilControl\cite{li12} is a distributed scheme for Sybil prevention using
  computational puzzles. Every node is supposed to periodically generate a
  puzzle for itself based on puzzles of its neighbors and solve it. Such puzzle
  is verifiable by any node using novel multi-hop verification and serves as a
  proof-of-work. Nodes that can not present an up-to-date proof-of-work are
  suspected of being a Sybil identity and removed from routing tables. This
  effectively bounds a number of Sybil identities an attacker can create.

  Computation puzzles have the advantage of being flexible. They can be done by
  any node, provided it has enough resources and can also scale their
  difficulty. It is also much harder for an attacker to fake solving them or use
  social engineering like in social network solutions.

\section{Other forms of attack}
  Eclipse and Sybil attacks do not usually constitute the goal of an attack, but
  are used to facilitate it. Such goal are usually to deny access to a resource
  or to fake it.

  Attacks that aim to fake possession of requested resource are best countered
  with cryptographic primitives. Using signatures, self-certifying data, or
  Merkle hash trees, like in BitTorrent, is an effective and efficient
  deterrent. To protect against malicious nodes dropping data or routing
  information the system needs to use replication and redundant routing. 

  Replication against malicious opponent is trickier than without her. There are
  two main strategies for replication: Place given resource at nodes with key
  close to the key of the resource or spread it over the network \cite[p.
  38]{urd11}. The first strategy allows for more efficient retrieval in DHTs
  such as Kademlia or Pastry, but is also more vulnerable to localized eclipse
  attack. Both strategies are easy to attack if an attacker has ability to
  choose her identifier arbitrarily and therefore a secure DHT requires secure
  node id assignment.

\section{Examples of real attacks}
These are not just theoretical divagations, but discussed attacks are really
used in practice and are effective.

\paragraph{Hijacking attack}
\cite{wan08} describes an attack on Kad network as implemented by eMule
file-sharing application. This is a version of a routing table poisoning attack
in which the attacker abuses lack of authentication and admission of control
which allows any node to advertise as any other. The attack is split into two
phases:
\begin{enumerate}
  \item In the preparation phase the attacker sets up N uniformly distributed
    virtual nodes which gather routing table information of their neighbors.
  \item In the attack phase each attacking node send to its neighbors bogus
    updates of the form \texttt{(ID\_B, IP\_A)}, where \texttt{ID\_B} is the key
    of existing victim node present in the recipient's routing table and
    \texttt{IP\_A} is the attacker's IP address.
\end{enumerate}

Authors show that with just 100Mbps bandwidth they were able to successfully
deny 80\% of keyword queries after 1-hour long preparation phase with a
network with 16000 nodes. Unfortunately, for the attacker, this cost is ongoing
since the attacking node needs to be responsive.

Authors also provide a version of this attack where, instead of providing
attacker's IP, the IP of the attacked node is used. Emule at the time did not
provide any check against this behavior so it would be effective in blocking
any effective look up. They also calculate that in this version it would take
around 200Mbps of network speed to attack the entire 1 million eMule network
with the result of blocking up to 75\% queries.

More recent versions of eMule contain basic protections against attacks, such
as \cite{tim11}:
\begin{description}
  \item{\textbf{Flood protection}} The number of messages received from one IP
    in a given time-frame is limited.
  \item{\textbf{IP limitations}} Only one node with given IP can be present in
    the routing table.
  \item{\textbf{IP verification}} Before a contact is added a three-way
    handshake is performed.
\end{description}

While those protections defend against small and unsophisticated attack it must
be noted that the Kad network is still vulnerable to moderately determined and
resourceful attacker.

\paragraph{Feasibility of Sybil attack}
Although Sybil attack is more expensive than above mentioned Hijacking attack
\cite{tim11} has shown that in Mainline DHT implementation of Kademlia used in
BitTorrent it is still easy. In their experiment they were able to fill
neighbors routing table with 160 Sybil nodes within 5 minutes after the attack
phase has begun. While similar attack would be impossible in newer versions of
Kad the only limitation is the fact that only one IP address is used.

\paragraph{Measurements of present attacks}
It is very easy to launch a Sybil attack in Mainline DHT and such attacks have
distinctive footprint. Wang \cite{wan12} has set up honeypots in the Mainline
DHT to detect and analyze possible Sybil attackers. He discovered an attacker
using a hijacking method described earlier running on Amazon's cloud who
maintains over 300000 thousand Sybil nodes. This attacker didn't seem to do much
and might have been just monitoring the network. A different attacker was
impersonating specific keyspace ranges and had distinctive IP address belonging
to international ISP service. He was redirecting queries from inside its domain
to other nodes inside it, localizing the traffic.

Authors conclude that although those attacks didn't seem to do much global harm,
the easiness with which they are executed proves that there are no costly
barriers to launching a successful attack to disable majority of DHT queries.

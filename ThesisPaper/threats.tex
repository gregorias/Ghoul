\chapter{DHT, its weaknesses, and defense mechanisms}
\section{DHT overview}
  [TODO write a section explaining what is DHT and how it works. Also
    explain Kademlia, Chord and Pastry]

  Distributed hash table is a type of peer-to-peer system providing topology
  creation and maintainance protocols. They are typically designed for and
  tightly coupled with data storage protocols for key-value pairs, hence the
  hash-table part. DHTs suffer from various vulnerabilities, which mostly stem
  from two diffuclties:
  \begin{enumerate}
    \item Node identity is a virtual concept and lacks connection to the real
      world. DHTs as a distributed storage space rely on data and message
      redundancy and cohesiveness of the topology. Both are easily broken if an
      attacker may either arbitrarily place itself in the topology (Eclipse
      attack) or get as many nodes as it wants (Sybil attack).

    \item Each node only maintains a local view over the network, which does not
      provide enough context to detect and prevent any malicious behaviour.
  \end{enumerate}

\section{Eclipse attack}
  Attacks on DHT can have 3 different goals:
  denial of service - the attacker aims to either stop the entire system from
  functioning or deny access from or to a part of it,
  subversion - the attacker wants to provide false data to other nodes or
  control them,
  information - the attacker wants to gather information about some nodes
  activity.

  Those goals are greatly facilitated if an attacker controls the routing tables
  of other nodes. An attack which poisons routing tables of other nodes by
  disproportionately filling them with entries to colluding malicious nodes is
  called an eclipse attack. The name comes from one of its effects, if a node's
  routing table is filled with malicious nodes it is effectively partitioned
  from the rest of the network.

  \subsection{Prevention mechanisms}

  Sit and Morris \cite{sit02} have been among the first to discuss general
  guidelines against preventing attacks in DHT networks. They advise desinging
  the system with verifiable constrainst so that a node can distinguish
  malicious behaviour from honest one and provide a proof to other nodes.
  Mechanisms, mainly data and node look up, should have a built-in redundancy so
  that a malicious node presence can be avoided in at least one path of
  execution. The lookup itself should be interactive and allow for clear view of
  progress, which allows a node to verify its correctness and initiate
  redundancy protocols if necessary. Finally, as in any distributed system,
  points of single responsibility should be avoided for obvious reasons.
  
  \subsubsection{Constrained routing table}
  Castro et al. \cite[p. 20]{urd11} have modified Pastry with additional
  constrained routing table and back up routing mechanism to protecct against
  eclipse attacks. If node key's assignment is truly random and routing table
  entries are narrowly specified then it is difficult for an attacker to inject
  malicious entries. DHTs such as Kademlia and Pastry are therefore more
  vulnerable than Chord.

  \subsubsection{Node scrambling}
  A different method to mitigate an eclipse attack is to constantly move node's
  ids in unpredictable ways so that it is infeasable to eclipse a local
  keyspace. Such solution is proposed by Condie et al. \cite[p. 21]{urd11}.
  Condie shows that a periodical random, impredictable node reassignment
  augmented by techniques which mitigate negative churn effects, that is
  grouping nodes into different timeout groups so that timeouts are not
  synchronized and precalculation of future routing table, significantly
  improves Castro's solution.

  Awerbach and Scheideler \cite{awe10}, while proposing a robust, distributed
  random number generator propose a scheme of global node rearrengment on each
  join node. In such a scheme once a join node they show that using a robust
  cuckoo rule to move $O(\log n)$ nodes makes the network resilient to
  join-leave attacks.

  \subsubsection{Topology graph analysis}
  Singh et al. \cite[p. 23]{urd11} observe that if malicious nodes form only a
  fraction of the network then during an Eclipse attack their in-degree must be
  higher than average in-degree of honest nodes. Therefore they propose a
  protocol which limits in-degree of nodes. Each node $x$ maintains an
  additional backpointer table, which contains nodes having $x$ in their routing
  tables. Every node $y$ periodically checks backpointer tables of their
  neighbours and if they are larger than a global constant or does not contain
  $y$ then it removes $x$ from its routing table.

  This scheme requires the backpointer check to be anonynomous, therefore the
  Scheme uses a common technique of onion routing using anonymizer network to
  achieve that. Anonymizer are not assumed to be honest and therefore it
  possible that a honest node will be labeled as malicious with small
  probability.

  This scheme is effective at limiting eclipse attacks, but it requires the
  global in-degree bound to be tight, especially for large networks, which may
  decrease the efficiency of base protocol.

\section{Sybil attack}
  DHTs routing mechanisms rely on the assumption that nodes are placed uniformly
  inside its key space and only a fraction of the network may fail, either due
  to churn or node failure. In case of eclipse attack or routing attack
  prevention mechanism they additionally assume that only a given portion of the
  network may be malicious. Those assumptions can be easily violated in an open
  DHT.

  Firstly, key generation is not verifiable. There is no way for a node to
  verify whether given key has been generated randomly. It is possible to use a
  statistical test to verify uniformity constraints, but they are often not
  enough to accuse specific node of malicious behaviour.

  Secondly, DHTs operate on an ephemeral sense of identity represented by a key.
  A Sybil attack happens when an attacker uses a single physical computer to
  represent multiple DHT nodes increasing his power without incurring additional
  cost. Since most protocols rely on honest nodes having a majority, a Sybil
  attack can shut down the system by even a modest attacker.

  Preventing a Sybil attack requires a gatekeeper or, more accurately, a
  bouncer. A protocol which prevents a single physical node to enter more than
  once and a way to throw out offenders. Achieving that in a fully distributed
  way is difficult. It is not possible to achieve adequate security if nodes
  only have access to their immediate surroundings. Such a limited view lacks
  necessary information to distinguish between bening and malicious behaviour.
  Additionally it's much easier to simply eclipse a node. Protocols that
  overcome this obstacle by sharing information globally do so at the cost of
  performance and scalibility. Such cost usually makes a DHT unpractical for
  larger networks. That's why most practical solutions make use of a central
  authority. This authority serves as a trusted entry point to the network, it
  generates signed random keys for newcomers and checks global constraints.

  Central authority has well-known drawbacks. Most of them stem from the fact
  that such authority represents a single point of failure. A failure of this
  authority may cause shut down of the network. Compromised authority is a
  security risk. It also acts negatively on scalability. Apart from
  technological disadvantages central authority is often an organizational
  problem. It forces an agent to maintain it in order to provide a service, this
  is done reluctantatly since it makes the agent responsible for maintaining the
  service and also Peer-to-Peer user community may be distrustful over
  centralized solution.

\subsection{Prevention mechanisms}

  The conclusion that only central authority solutions are practical for
  prevention of Sybil attacks was first deduced by \cite{dou02}, which also is
  the first paper studying this class of attacks \cite[p. 5]{urd11}

  [Tutaj jest praktycznie to co w \cite{urd11}]
  \subsubsection{Centralized gatekeeper}
  Castro et. al. \cite{cas02} agrees with \cite{dou02} conclusion and proposes
  using a set of central trusted authorities which certify random key and public
  key pair. They suggest adding an IP address to the certificate, so that the
  certificate would be binded to physical host. Unfortunately this solution is
  known to be problematic. First of all IP address can be spoofed and with IPv6
  privacy extensions changed al together. Second of all it would cause a problem
  for users behind NAT.

  Centralized gatekeeper gives much flexibility in implementing an
  authentication and sybil attack prevention scheme. For example there may be
  required a small monetary payment for entry into the DHT network or a
  computational puzzle may have to solved.

  \subsubsection{Distinction based on network footprint}

  Without centralized authority network peers would have to register new nodes
  in a distributed way. To distinguish distinct identities from same ones they
  may use network characterestics. Such characteristic are usually easy to check
  in the sense that any node may do it, giving low barrier of entry to the
  network, but it carries an assumption that they are unfalsifiable, which is
  not always the case.

  A simple ID calculation based on IP address and port number is proposed by
  Dinger and Hartensteing \cite[p. 6]{dou02}. They propose that a node would
  register $r$ Chord registration nodes whose position is calculated from hash
  of node's ID. If majority of registration nodes accept the new node then it is
  said to be accepted. A node registration fails if number nodes with given IP
  address exceeds a constant $a$. This solution requires all nodes to confirm
  validity of their neighbours.

  IP addresses may be spoofed and cause a problem for NAT users. Wang et. al.
  \cite[p. 7]{dou02} proposed adding router's IP and MAC address and RTT
  measurements between the node a designated set of landmarks. A similar idea
  may be found in Bazzi and Konjevod \cite[p. 7]{dou02} where nodes are put into
  $d$-dimensional space based on physical location of a node and a node are said
  to be distinct if their coordinates are far away enough.

  \subsubsection{Distinction based on social network}

  Another class of defenses uses a social network, a graph of trust
  relationships established by humans, to discover or mitigate a Sybil attack.
  In such defenses it is assumed that each node can verifiably identify itself
  as a node in the social network and that each edge forms a trust relationship
  which means that two people are friends and that they trust each other not to
  launch a Sybil attack. If any node was to launch a Sybil attack then the
  social network graph will exhibit special properties which may be detected. 

  Although such schemes show good performance, both in terms of application
  overhead and detectability, they are difficult to apply in real world
  applications.  [TODO aplikacja musi skądś ten graf wziąć co często jest
  niemile widziane lub niepraktyczne]

  SybilGuard \cite{hai06} mitigates effects of a Sybil attack by limiting the
  number of sybil nodes and recognized sybil groups. It defines the edges
  between honest nodes and sybil nodes as attack edges and notes that in a
  social graph there can be only few such edges. Then it allows a node to
  partition its neighbours into classes and guarantees with high probability
  that a number of classes that contain a Sybil node is limited by the number of
  attack edges. Although the scheme does not detect Sybil nodes it gives a
  guarantee that with a sufficient redundandancy an attacker will not be able to
  intercept a query or manipulate data.

  \subsubsection{Puzzles}
  SybilControl\cite{li12} is a distributed scheme for sybil prevention using
  computational puzzles. Every node is supposed to periodically generate a
  puzzle for itself based on puzzles of its neighbours and solve. Such puzzle
  is verifiable by any node using novel multi-hop verification and serves as a
  proof-of-work. Nodes that can not present an up-to-date proof-of-work are
  suspected of being a Sybil identity and removed from routing tables. This
  effectively bounds a number of sybil identities an attacker can create.

  Computation puzzles have the advantage of being flexible. They can be done by
  any node, provided it has enough resources and can also scale their
  difficulty. It is also much harder for an attacker to fake solving them or use
  social engineering like in social network solutions.

\section{Discussion}
  Eclipse and Sybil attacks do not usually constitute the goal of an attack, but
  are used to facilitate it. Such goal are usually to deny access to a resource
  or to fake it.

  Attacks that aim to fake possession of requested resource are best counter
  with cryptography. Using signatures, self-certyfying data, or Merkle hash
  trees, like in bittorrent, is an effective and efficient deterrent.
  To protect against malicious nodes dropping data or routing information the
  system needs to use replication and redundant routing. 

  Replication against malicious opponent is trickier than with her. There are
  two main strategies for replication: Place given resource at nodes with key
  close to the key of the resource or spread it over the network \cite[p.
  38]{urd11}. The first strategies allows for more efficient retrieval in DHTs
  such as Kademlia or Pastry, but is also more vulnerable to localized eclipse
  attack. Both strategies are easy to attack if an attacker has ability to
  choose her identifier arbitrarily and therefore a secure DHT requires secure
  node id assignment.

\section{Examples of real attacks}
These are not just theoretical divagations, but discussed attacks are really
used in practice:

[TODO attacks on Vanish, Maze, Kad (Poisoning kad network)]

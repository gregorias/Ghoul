\chapter{Conclusion and future work}
\label{ch:conclusion}
We have presented Ghoul - a security extension of Kademlia distributed hash table protocol.
We also provide a prototype implementation of Ghoul, i.e. message extensions,
certificate generation and management, and registrars.
This prototype allows for deployment of Kademlia DHT with centralized
gatekeeper, where node's entry is controlled and key-generation is guaranteed to
be random.

The main goal of the prototype is to show that it is possible to achieve secure
DHT without additional effort from the user.
We believe that the example shown in Chapter \ref{ch:implementation} proves
that, indeed, running Ghoul is as simple as running a bare Kademlia node.

In the future we plan to finish the implementation with SybilControl protocol
and release a non-prototype version of the library publicly.

Additionally, we have presented the dfuntest's design pattern and implementation
for writing distributed tests with centralized control.
Dfuntest offers a coherent and expressive abstraction for distributed testing.
This abstraction allows for clean automation of the testing process which in
turn also gives greater control of the real-world testing environment.

The main goal of dfuntest are jUnit-like acceptance tests: set up environments,
run the tested application, check some assertions on the state. However, thanks
to flexible test scripts, we are currently using dfuntest also for performance
evaluation of various subsystems of
nebulostore\footnote{\url{http://nebulostore.org}}, which previously required
ad-hoc deployment scripts.

In the immediate future we plan to extend dfuntest library with ability to run
tests of heterogeneous applications in heterogeneous environmnents---e.g., a
client-server application where capabilities of environments are different.
We also plan better mechanisms for coping with failing environments: a failure
of remote host, independent from the application. Currently such failure can
only be detected with ssh queries.

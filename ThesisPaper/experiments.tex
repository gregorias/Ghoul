\chapter{Experimental evaluation}
\label{ch:evaluation}

We use dfuntest to test and verify correctness of our implementation
(Section~\ref{sec:exampl-test-kademl}): in particular, we show how to verify
that the implementation maintains the correct DHT topology.

Dfuntest was initially built for our bare Kademlia implementation.
Ghoul extends Kademlia with additional background security protocols, therefore distributed tests for Kademlia are also tests for Ghoul (with small adjusments in environment preparation).
For this historical reason the name Kademlia is used in examples in
Section~\ref{sec:exampl-test-kademl}.
\section{Testing Ghoul DHT with dfuntest}
\label{sec:exampl-test-kademl}
In this section we will present how we have prepared dfuntests for Ghoul.

\subsection{Preparation}
We need to provide to dfuntest information on how to start and use our
application. This is done only once and does not need to be rewritten as long as
the Ghoul API and its requirements remain the same.

\paragraph{\texttt{App}} 
\texttt{KademliaApp} extends the proxy \texttt{App} class with Ghoul's interface methods.
\texttt{KademliaApp}'s main responsibility is to translate Java method calls into the RPC-over-HTTP interface of the Ghoul application.
An example code of those methods is shown in \ref{fig:app_example}.
\texttt{KademliaApp} is parametrized by the base \texttt{Environment}, as we use only standard operations.
To construct, \texttt{KademliaApp} requires the URI address of the external
application's interface; and the environment on which it works.

\begin{figure}[tbp]
\begin{lstlisting}
public synchronized void startUp() throws IOException {
  List<String> runCommand = new LinkedList<>();
  runCommand.add(mJavaCommand);
  runCommand.add("-Dorg.slf4j.simpleLogger.logFile=" + LOG_FILE);
  runCommand.add("-Dorg.slf4j.simpleLogger.defaultLogLevel=trace");
  runCommand.add("-cp");
  runCommand.add("lib/*:Kademlia.jar");
  runCommand.add("me.gregorias.Kademlia.interfaces.Main");
  runCommand.add("Kademlia.xml");
  mProcess = mKademliaEnv.runCommandAsynchronously(runCommand);
}

public Collection<NodeInfo> findNodes(Key key) throws IOException {
  Client client = ClientBuilder.newClient();
  WebTarget target = client.target(mUri).path("find_nodes/" + key.toInt());

  NodeInfoCollectionBean beanColl;
  try {
    beanColl = target.request(MediaType.APPLICATION_JSON_TYPE)
        .get(NodeInfoCollectionBean.class);
  } catch (ProcessingException e) {
    throw new IOException("Could not find node.", e);
  }
  NodeInfoBean[] beans = beanColl.getNodeInfo();
  Collection<NodeInfo> infos = new LinkedList<>();
  for (NodeInfoBean bean : beans) {
    infos.add(bean.toNodeInfo());
  }
  return infos;
}
\end{lstlisting}
\caption{\texttt{KademliaApp}: start a remote Ghoul instance; and use
HTTP-RPC to find neighboring nodes of the instance.}
\label{fig:app_example}
\end{figure}

\paragraph{\texttt{KademliaEnvironmentPreparator}} (Figure~\ref{fig:prepare})
\texttt{KademliaEnvironmentPreparator} copies Kademlia's dependency jar files and configuration
files to to the target environment.

\texttt{KademliaEnvironmentPreparator} uses methods of \texttt{Environment} to deploy the application.
graph;
\begin{figure}[tbp]
\begin{lstlisting}
public void prepare(Collection<Environment> envs)
    throws IOException {
  Collection<Environment> preparedEnvs = new LinkedList<>();
  Environment firstEnv = findFirstEnvironment(envs);
  for (Environment env : envs) {
    XMLConfiguration xmlConfig = prepareXMLConfiguration(env, firstEnv);
    try {
      xmlConfig.save(XML_CONFIG_FILENAME);
      Path localConfigPath = FileSystems.getDefault().getPath(LOCAL_CONFIG_PATH).toAbsolutePath();
      env.copyFilesFromLocalDisk(localConfigPath, ".");
      env.copyFilesFromLocalDisk(LOCAL_JAR_PATH.toAbsolutePath(), ".");
      env.copyFilesFromLocalDisk(LOCAL_LIBS_PATH.toAbsolutePath(), ".");
      preparedEnvs.add(env);
    } catch (ConfigurationException | IOException e) {
      clean(preparedEnvs);
      throw new IOException(e);
    }
  }
}
\end{lstlisting}
\caption{\texttt{KademliaEnvironmentPreparator} (fragment): prepare remote environments by copying jars and configuration files. Methods \texttt{restore}, \texttt{collectOutput}, \texttt{cleanOutput} and \texttt{cleanAll} are similar.}
\label{fig:prepare}
\end{figure}


\paragraph{Main and \texttt{KademliaAppFactory}} To glue those we also need to
define a simple factory which instantiates \texttt{KademliaApp}s given prepared
\texttt{Environments}. Also the entry point to dfuntest application - the main
method - needs to be defined.

\subsection{Test script}

We show how to check whether the topology graph induced by Kademlia routing
tables is connected (consistent).

To define the testing scenario, the tester writes the \texttt{run} method defined by the
\texttt{TestScript} interface (Figure~\ref{fig:core}).

\begin{figure}[tbp]
\begin{lstlisting}
public TestResult run(Collection<KademliaApp> apps) {
  try {
    startUpApps(apps); // Call startUp method of each KademliaApp.
    Thread.sleep(START_UP_DELAY_UNIT.toMillis(START_UP_DELAY));
  } catch (IOException e) {
    return new TestResult(Type.FAILURE,
        "Could not start up applications.", e);
  }
  try {
    startKademlias(apps); // Send start HTTP-RPC to each interface
  } catch (KademliaException | IOException e) {
    shutDownApps(apps);
    return new TestResult(Type.FAILURE, "Could not start Kademlias.", e);
  }
  mResult = new TestResult(Type.SUCCESS,
      "Connection graph was consistent the entire time.");
  scheduleCheckerToRunPeriodically(new ConsistencyChecker(apps))
  waitTillCheckerFinished();
  stopKademlias(apps);
  shutDownApps(apps);
  return mResult;
}

private class ConsistencyChecker implements Runnable {
  @Override
  public void run() {
    Map<Key, Collection<Key>> graph;
    try {
      // Call GET_ROUTING_TABLE RPC of each node
      graph = getConnectionGraph(mApps);
      ConsistencyResult result = checkConsistency(graph);
      if (result.getType() == ConsistencyResult.Type.INCONSISTENT) {
        mResult = new TestResult(Type.FAILURE,
            "Graph is not consistent starting from: " + result.getStartVert()
            + " could only reach " + result.getReachableVerts().size());
        shutDown();
        return;
      }
    } catch (IOException e) {
      mResult = new TestResult(Type.FAILURE,
          "Could not get connection graph: " + e + ".");
      shutDown();
      return;
    }
    --mCheckCount;
    if (mCheckCount == 0) {
      shutDown();
    }
  }
  ...
}
\end{lstlisting}
\caption{\texttt{KadmliaConsistencyTestScript} (fragment) periodically checks consistency of the induced Kademlia graph.}
\label{fig:core}
\end{figure}

\subsection{Dfuntest validation tests}

We configured Ghoul build system to create a separate .jar package with the above dfuntest.
The tests are executed as a standard Java application.
The main method expects an XML configuration file as its argument.
This configuration file contains all pertinent parameters for setting up
environments and controlling test execution.
For example to start a test suite on 7 remote hosts we provide an XML
configuration file with following information:

\begin{verbatim}
<SSHEnvironmentFactory>
  <hosts>
    roti.mimuw.edu.pl,
    prata.mimuw.edu.pl,
    planetlab2.wiwi.hu-berlin.de,
    planetlab2.s3.kth.se,
    planetlab1.u-strasbg.fr,
    planetlab01.tkn.tu-berlin.de,
    pl1.uni-rostock.de
  </hosts>
  <username>
    mimuw_user
  </username>
  <privateKeyPath>
    /home/mimuw_user/.ssh/id_rsa
  </privateKeyPath>
</SSHEnvironmentFactory>
\end{verbatim}

The rest is handled by the dfuntest application and at the end a human-readable
report directory is produced. This report directory contains a separate
directory for each executed TestScript test with logs and summary report.
Additionally there's a summary for all the executed tests
(Figure~\ref{fig:sumrep}).

\begin{figure}[tbp]
\begin{verbatim}
[SUCCESS] KademliaChurnConsistencyCheckTestScript
[SUCCESS] KademliaConsistencyCheckTestScript
[SUCCESS] KademliaDataReplicationTestScript
\end{verbatim}
\caption{A summary report for the test in which all \texttt{TestScript}s passed}
\label{fig:sumrep}
\end{figure}

We have already explained \texttt{KademliaConsistencyCheckTestScript}. 
A variation of that test is:
\texttt{KademliaChurnConsistencyCheckTestScript}. 
The churn version additionally simulates churn in-between consistency checks by
restarting nodes with 50\% probability.\\
\texttt{KademliaDataReplicationTestScript} makes each node put a key-value pair
into distributed storage and checks whether on \texttt{GET} request the data is
replicated.

Because a successful result is uninteresting we'll show what happens on failure.
At the same time we verify that our tests actually test and not just display
success.
To inject a failure into Ghoul, we used a too small bucket size (1, compared
with a usual value of 5).
Such a small bucket should make the graph inconsistent, because node finding
messages will have too little diversity.
This set up led to failure of all tests (occassionally data replication works
fine in that scenario).
The summary for all tests (\ref{fig:sumrep_failure}) shows failure in all
tests.
The summary for consistency check (Figure~\ref{fig:conrep}) clearly points to a
misbehaving node (19). 
This points the tester where to look to the source of this error.

\begin{figure}[tbp]
\begin{verbatim}
[FAILURE] KademliaChurnConsistencyCheckTestScript
[FAILURE] KademliaConsistencyCheckTestScript
[FAILURE] KademliaDataReplicationTestScript
\end{verbatim}
\caption{A summary report for the test in which all \texttt{TestScript}s have
failed}
\label{fig:sumrep_failure}
\end{figure}

\begin{figure}[tbp]
\begin{verbatim}
[FAILURE] Found inconsistent graph.
The connection graph was:
18: [16, 20, 24, 0]
19: [16, 20, 24, 0]
16: [17, 18, 20, 24, 0]
17: [16, 18, 20, 24, 0]
22: [23, 20, 16, 24, 0]
23: [22, 20, 16, 24, 0]
20: [21, 23, 16, 24, 0]
21: [20, 23, 16, 24, 0]
24: [16, 0]
2: [3, 0, 4, 11, 16]
3: [2, 0, 4, 9, 16]
0: [1, 2, 4, 8, 16]
1: [0, 2, 4, 8, 16]
6: [7, 4, 0, 11, 16]
7: [6, 4, 0, 9, 16]
4: [5, 6, 0, 9, 16]
5: [4, 6, 0, 8, 16]
10: [8, 12, 0, 16]
11: [8, 12, 0, 16]
8: [9, 10, 12, 0, 16]
9: [8, 10, 12, 0, 16]
14: [15, 12, 8, 0, 16]
15: [14, 12, 8, 0, 16]
12: [13, 14, 8, 0, 16]
13: [12, 14, 8, 0, 16]

It's strongly connected components are:
[19]
[18, 16, 17, 22, 23, 20, 21, 24, 2, 3, 0, 1, 6, 7, 4, 5, 10, 11, 8, 9, 14, 15, 12, 13]
\end{verbatim}
\caption{A report generated after KademliaConsistencyCheckTestScript fails.}
\label{fig:conrep}
\end{figure}

In case an error requires a debugging process the developer may easily change
the testing environment and test parameters by changing the configuration file
to facilitate the bug triage.

\section{Performance evaluation}
\label{sec:performance}

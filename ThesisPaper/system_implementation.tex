\chapter{Ghoul implementation}
In this chapter we will present the Ghoul's implementation.
We will focus on general techniques used in Ghoul's development to produce high-quality code.
In section \ref{sec:build} we present build tools used and how Ghoul may be run.
Ghoul is an open-source project.
An up-to-do Ghoul's code is available for public audience at \url{https://github.com/gregorias/ghoul}.

\section{Build system and running}
\label{sec:build}

\subsection{Gradle}
Currently Ghoul's code contains around 8000 lines of code in 126 Java files (excluding dfuntest's project).
To maintain this amount of code we use Gradle \footnote{\url{https://gradle.org/}} build automation framework.
In Gradle, developers write configuration scripts called \texttt{build.gradle} in which they define code's properties --- e.g. name of the project, project's dependencies; and build tasks.
Those configuration scripts are written in a functional, dynamically typed language.
This allows for expressive and succint definition of build tasks, which is an
advantage over more traditional build framework - Maven \footnote{\url{https://maven.apache.org/}}.
Next, Gradle performs, using provided information, typical task in Java's build ecosystem: compilation, packaging, dependency resolution, unit testing, software validation etc.
Gradle allows developers to download and build Ghoul in two steps: Download the software from GitHub and run build task in gradle (shown in figure \ref{fig:ghoul_build_process})

\begin{figure}[tb]
\begin{verbatim}
> git clone https://github.com/gregorias/ghoul
...

> cd ghoul

> ./gradlew build
:assemble UP-TO-DATE
:check UP-TO-DATE
:build UP-TO-DATE
:ghoul-core:compileJava
:ghoul-core:processResources UP-TO-DATE
:ghoul-core:classes
:ghoul-core:jar
:ghoul-core:javadoc
...
:ghoul-dfuntest:pmdMain
:ghoul-dfuntest:pmdTest UP-TO-DATE
:ghoul-dfuntest:test UP-TO-DATE
:ghoul-dfuntest:check
:ghoul-dfuntest:build

BUILD SUCCESSFUL

Total time: 31.617 secs
\end{verbatim}
\caption{Ghoul's build process}
\label{fig:ghoul_build_process}
\end{figure}

We have added additional automatic code quality checking tools to Ghoul's automatic build:

\begin{description}
  \item{\textbf{jUnit}\footnote{\url{http://junit.org}}} 
    jUnit is the framework for writing repeatable unit tests for Java code.
  \item{\textbf{JaCoCo}\footnote{\url{www.eclemma.org/jacoco/}}}
    JaCoCo is a code coverage library which automatically analyzes code executed by jUnit tests and produces an HTML or XML report summarizing code coverage.

  \item{\textbf{PMD}\footnote{\url{http://pmd.sourceforge.net}}}
    PMD is a source code analyzer.
    It finds common programming flaws like unused variables, empty catch blocks, unnecessary object creation etc.
  \item{\textbf{Findbugs}\footnote{\url{http://findbugs.sourceforge.net/}}}
    Findbugs is a static analysis tools which looks for bugs in Java's code.
    It can detect such bugs are code smells as: unused variables, null value dereference, use of \texttt{AtomicBoolean} for comparison etc.
    
\end{description}

We haven't included dfuntest's test into the automatic build, because this is a long task.
However dfuntest's testing suite is always run manually to confirm code's correctness.

\subsection{Usage}

Although Ghoul is mainly designed to be used as a library, it is also possible to run it as a standalone application.
In standalone mode Ghoul additionally exposes an RPC-over-HTTP interface over which it is possible to query and manipulate the DHT node (shown in table \ref{tab:http_rpc}).

An example session which runs 2 registrars and 3 DHT nodes is shown in figures 
\ref{fig:ghoul_manual_run} and \ref{fig:ghoul_manipulation}.
In the session we use curl to send RPC-over-HTTP queries to DHT nodes.
First we start registrars and then DHT nodes.
Afterwards we get routing tables of the first and third DHT node.
Then we put \texttt{DATA} string under the key \texttt{100}.
Later we ask the third node get all replicas of the data under key \texttt{100}.

\begin{figure}[tb]
\begin{verbatim}o
> java -cp ghoul-core-0.1.jar:lib/* me.gregorias.ghoul.interfaces.RegistrarMain\
  registrar0.xml > /dev/null &

> java -cp ghoul-core-0.1.jar:lib/* me.gregorias.ghoul.interfaces.RegistrarMain\
  registrar1.xml > /dev/null &

> java -cp ghoul-core-0.1.jar:lib/* me.gregorias.ghoul.interfaces.Main \ 
  kademlia0.xml >/dev/null &

> java -cp ghoul-core-0.1.jar:lib/* me.gregorias.ghoul.interfaces.Main \ 
  kademlia1.xml >/dev/null &

> java -cp ghoul-core-0.1.jar:lib/* me.gregorias.ghoul.interfaces.Main \ 
  kademlia2.xml >/dev/null &

> curl -X POST 127.0.0.1:10000/start

> curl -X POST 127.0.0.1:10001/start

> curl -X POST 127.0.0.1:10002/start
\end{verbatim}
\caption{Ghoul's starting commands}
\label{fig:ghoul_manual_run}
\end{figure}

\begin{figure}[tb]
\begin{verbatim}
> curl -X GET 127.0.0.1:10000/get_routing_table
{"nodeInfo":[{"inetAddress":"localhost","port":9001,
    "key":"fca426586c9a211513c4338e975c5508250c1a0"},
{"inetAddress":"localhost","port":9002,
    "key":"2d1386c9c601676189faa7790b60dbeda3ed3e9b"}]}

> curl -X GET 127.0.0.1:10001/get_routing_table
{"nodeInfo":[{"inetAddress":"localhost","port":9000,
    "key":"15624f385826cae60bef58da5b53b4dc1cf070bd"},
{"inetAddress":"localhost","port":9002,
    "key":"2d1386c9c601676189faa7790b60dbeda3ed3e9b"}]}

> curl -X POST --header "Content-Type:application/octet-stream" \ 
--data-binary DATA 127.0.0.1:10000/put/100

> curl -X GET 127.0.0.1:10002/get/100
["REFUQQ==","REFUQQ==","REFUQQ=="]
\end{verbatim}
\caption{Ghoul's status query and DHT manipulation}
\label{fig:ghoul_manipulation}
\end{figure}

\section{Implementation}

%Ghoul's class design uses SOLID [Agile Software Development, Principles,
%Patterns and Practices] principles of object design.
%SOLID principles are a set of 5 guidelines for designing objects such that the entire code will more likely to be easier to maintain and extend.
%Those 5 principles are:

%\begin{description}
  %\item{\textbf{Single responsibility principle}}
%\end{description}
%
%SOLID principles increase code's modularity and reusability.
%Their effects are especially visible in unit tests.
%For instance the core class - \texttt{KademliaRoutingImpl}, implements the core kademlia protocol with security checks, takes all its dependencies as constructor parameter (DependencyInversion).
%Its constructor's header is show on figure \ref{fig:routing_constr_header}.
%Large number of parameters is sometimes considered a code smell, but dependency inversion is a more important consideration in this case and additionally a builder design pattern is used for creating instances of \texttt{KademliaRouting}.
%The dependencies of \texttt{KademliaRoutingImpl} are mostly interfaces (apart from \texttt{Key}, \texttt{TimeUnit}).
%This is a realization of the Liskov substitution principle; Later we will that this allows creation of mocks and stubs in unit-testing for testing not-trivial scenarios.
%
%\begin{figure}[tbp]
%\begin{lstlisting}
%KademliaRoutingImpl(Key localKey,
                    %NetworkAddressDiscovery networkAddressDiscovery,
                    %MessageSender sender,
                    %ListeningService listeningService,
                    %int bucketSize,
                    %int alpha,
                    %Collection<NodeInfo> initialKnownPeers,
                    %long messageTimeout,
                    %TimeUnit messageTimeoutUnit,
                    %long heartBeatDelay,
                    %TimeUnit heartBeatDelayUnit,
                    %PersonalCertificateManager personalCertificateManager,
                    %CertificateStorage certificateStorage,
                    %PrivateKey personalPrivateKey,
                    %ScheduledExecutorService scheduledExecutor,
                    %Random random) {
%\end{lstlisting}
%\caption{\texttt{KademliaRoutingImpl}}
%\label{fig:routing_constr_header}
%\end{figure}


\chapter{Introduction}

Peer-to-Peer applications are applications which run on multiple computers in
such a way that each instance is treated equally.
This is in direct opposition to a client-server model, such as is used in, for
example, web-browsing.
In client-server model client instances request data from the server, which
stores and serves the data.
Peer-to-Peer model is popular among applications that need to scale dynamically
with the size of their user-base and applications for which users want privacy
and independence from central authority.
A good example of both of those factors at play is visible in BitTorrent.
BitTorrent is a file-sharing protocol.
A BitTorrent application, which wants to download a file, acquires a list of
other computers that are interested in that file (either want to download it or
have it).
Next, the application downloads missing parts of the file from hosts that have
those parts, while at the same time distributes parts of the file it has to
other hosts.
This kind of sharing approach means that with the increase of an interest in a
file the total bandwidth of hosts having the file also increases.
Since there is no central-authority, no one may easily collect data about user
behavior or shutdown the service.

A fundamental part of a Peer-to-Peer application is how different instances
communicate with each other.
Unlike the client-server model, in which there is global known, always-present
authority to communicate to, Peer-to-Peer applications have to handle users
dynamically joining and leaving the network (this phenomenon is called churn)
and find a way to efficiently send messages between two arbitrary nodes.
This problem may be compared the problem of sending a message to an unknown
town.
We are in a town X and want to find out what towns are present in the country
and how to send a message to a different town.
However, we only know how to reach neighboring towns and we do not know the
entire map of the country.

One of the roles of a Peer-to-Peer application is to define a protocol for
establishing a neighborhood relation between two nodes (equivalent to a road in
our analogy; the entire map is called a topology) and establishment of routes
between two non-neighboring nodes.
A popular approach is to use structured topologies created by Distributed Hash
Tables (DHT).
A structural topology means that there are strict rules which nodes may be
neighbors and how should messages propagate.
Topology may also be unstructured, meaning lack of or soft rules.
The benefit of the structural approach is that the structure that they create
has certain properties that guarantee that routes between non-neighboring nodes
will be relatively short and easy to find.

Unfortunately, the structural nature of the topologies created by DHTs and
their Peer-to-Peer character make them vulnerable to attacks.
Since topology is the basis of a Peer-to-Peer application, a successful attack
on it is often enough to shut it down entirely.
One type of attack is a Sybil attack.
An adversary advertises itself as multiple, independent host (towns).
It makes us think that the network (country) is populous, but in reality every
message is sent to only one host.
Another is an Eclipse attack.
An honest host is surrounded by malicious hosts; whenever anyone wants to find
the honest host then their queries are blocked by the adversary.

Peer-to-Peer applications using DHTs often leave the DHT woefully unprotected.
We'll show that, for example, Bittorent's DHT - MainlineDHT - is especially
vulnerable to even a modest adversary.
We think there are two main reasons for this lack of care.
First, DHT security mechanisms are seen as unpractical, either they require
unreasonable resources (such as access to a social network), or have significant
performance overhead.
Second, most security mechanisms are not backward compatible with existing
implementations.
In this thesis we design Ghoul as the answer to the first objection.
We show that it is possible to have a reasonable DHT protocol, that is secure
against common types of attacks, does not require additional resources from the
user (such as access to his or her social network), and whose message complexity
of common operations is at worst logarithmic with regard to the size of the
network.
We also argue that some backward compatible security extensions, such as those
used in Kad DHT (a DHT used by eMule file-sharing application), are not
bullet-proof and with incoming wide adoption of IPv6, they might be rendered
void.

Additionally, to ensure high-quality of the prototype, we designed and
implemented a framework distributed testing of Peer-to-Peer applications -
dfuntest.
Dfuntest was originally developed by the author to test Kademlia application in
Planet-lab environment, but its design later proved to be flexible enough to be
a general framework for any P2P application.

\section{Overview}
In chapter \ref{ch:threats} we present the problem domain.
We define what is a Distributed Hash Table (DHT) and evaluate its typical
weaknesses.
We show what are the main classes of security threats that DHTs are vulnerable
to.
Finally we review proposed security mechanisms.

In chapter \ref{ch:description} we describe the protocols used in Ghouls and
discuss its strength and weaknesses.

In chapter \ref{ch:dfuntest} we present dfuntest - a framework for testing
distributed peer-to-peer applications.

In chapter \ref{ch:implementation} we explain how Ghoul is built and run as a
stand-alone application.
We focus on presenting tools and techniques used to guarantee high-quality code.

In chapter \ref{ch:conclusion} we summarize our contribution and state future
plans towards the prototype we have developed.

\section{Contributions}
This thesis provides the following contributions:
\begin{itemize}
  \item Design of Ghoul - a security extension of the Kademlia protocol that
    protects against common types of attacks.
  \item Ghoul's prototype - Java implementation of the Kademlia protocol with
    message extensions, certificate management, and registration mechanisms.
  \item Design and implementation of dfuntest - a framework for automation of
    distributed tests of Peer-to-Peer applications.
\end{itemize}

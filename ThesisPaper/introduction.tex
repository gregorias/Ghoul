\chapter{Introduction}

BitTorrent --- a popular file sharing protocol --- is used by millions of users
around the world, yet current implementations of its Distributed Hash Table
(DHT) protocol, which may be used to download file metadata, are insecure.
A modest attacker may abuse the Distributed Hash Table protocol to slow down the
metadata search, block access to some files, or even effectively shut down the
service for significant portion of its users.
Luckily for BitTorrent, DHT is only an optional mechanism for metadata
retrieval, but not every application may treat DHT as an optional mechanism.
DHT is a general and efficient tool that serves as a basis for Peer-to-Peer
applications.
A basis, which insecurity may break the entire application and which security is
hard to guarantee.

Peer-to-Peer (P2P) applications are applications which run on multiple computers
in such a way that each instance is treated equally.
P2P applications are in direct opposition to client-server applications, such as
used in, for example, web-browsing.
In client-server model client instances request data from the server, which
stores and serves the data.
Peer-to-Peer model is popular among applications that need to scale dynamically
with the size of their user-base without using application provider's resources
and applications for which users want privacy and independence from central
authority.
A good example of both of those factors at play is visible in BitTorrent.
A BitTorrent application, which wants to download a file, acquires a list of
other computers that are interested in that file (either want to download it or
have it).
Next, the application downloads missing parts of the file from hosts that have
those parts, while at the same time distributes parts of the file it has to
other hosts.
This kind of sharing approach means that with the increase of an interest in a
file the total bandwidth of hosts having the file also increases.
Since there is no central authority, no one may easily collect data about user
behavior (users do not have to authenticate themselves and may use anonymizing
tools, such as Tor) or shutdown the service.

A fundamental part of a Peer-to-Peer application is how different instances
communicate with each other.
Unlike the client-server model, in which there is a global known, always-present
authority to communicate to, Peer-to-Peer applications have to handle users
dynamically joining and leaving the network (this phenomenon is called churn)
and find a way to efficiently send messages between two arbitrary nodes.
This problem may be compared to the problem of sending a message to an unknown
town.
We are in a town and we want to find out what other towns are in the country and
how to send a message to those towns.
However, we only know how to reach neighboring towns and we do not know the
entire map of the country.
Our neighbors are in the same situation so we have to establish a cooperation
protocol to help each other achieve our goals.

One of the roles of a Peer-to-Peer application is to define this cooperation
protocol for establishment of a neighborhood relation between two nodes
(equivalent of a road in our analogy; the entire map is called a topology) and
establishment of routes between two non-neighboring nodes.
A popular approach is to use structured topologies created by DHT.
A structural topology means that there are strict rules which nodes may be
neighbors and how messages should propagate.
Topology may also be unstructured, meaning non-strict (or even lack of) rules.
The benefit of the structural approach is that the structure that they create
has certain properties that guarantee that routes between non-neighboring nodes
will be relatively short and easy to find.

Unfortunately, the structural nature of the topologies created by DHTs and
their Peer-to-Peer character make them vulnerable to attacks.
Since topology is the basis of a Peer-to-Peer application, a successful attack
on the topology is often enough to shut down the application entirely.
One type of attack is a Sybil attack.
An adversary advertises itself as multiple, independent hosts (towns).
It makes us think that the network (the country) is populous, but in reality
every message is sent to only one host.
The adversary's host then has larger control over the network than it should
have and may, for example, eavesdrop, poison (maliciously change its content),
or drop messages.
Another type of attack is an Eclipse attack.
An honest host is surrounded by malicious hosts; whenever anyone wants to find
the honest host then their queries are blocked by the adversary.

Peer-to-Peer applications using DHTs often leave the DHT woefully unprotected.
We'll show that, for example, BitTorent's DHT --- Mainline DHT --- is especially
vulnerable to even a modest adversary.
We think there are two main reasons for this lack of care.
First, DHT security mechanisms are seen as unpractical: either they require
unreasonable resources (such as access to a social network), or have significant
performance overhead.
Second, most security mechanisms are not backward compatible with existing
implementations.

In this thesis we design Ghoul --- a prototype secure DHT ---
as an answer to the impracticality concerns.
We show that it is possible to have a reasonable DHT protocol, that is secure
against common types of attacks, does not require additional resources from the
user (such as access to user's social network), and whose message complexity of
common operations is at worst logarithmic with regard to the size of the
network.
We also argue that some backward compatible security extensions, such as those
used in Kad DHT (a DHT used by eMule file-sharing application), are not
bullet-proof and with incoming wide adoption of IPv6, they might be rendered
void.

Additionally, to ensure high quality of the prototype, we designed and
implemented dfuntest --- a framework for distributed testing of Peer-to-Peer
applications.
Dfuntest was originally developed by the author to test Kademlia application in
Planet-lab environment, but its design later proved to be flexible enough to be
a general framework for any P2P application.

We hope that Ghoul will prove to be an useful tool to application developers in
creation of secure Peer-To-Peer applications.

\section{Outline}
In Chapter \ref{ch:threats} we present the problem domain.
We define what is a Distributed Hash Table (DHT) and evaluate its typical
weaknesses.
We discuss main types of security threats that DHTs are vulnerable to.
Finally, we review security mechanisms proposed in related work.
In Chapter \ref{ch:description} we describe the protocols used in Ghoul and
discuss Ghoul's strengths and weaknesses.
In Chapter \ref{ch:implementation} we explain how Ghoul is built and run as a
stand-alone application.
We focus on presenting tools and techniques used to guarantee high-quality code.
In Chapter \ref{ch:dfuntest} we present dfuntest --- a framework for testing
distributed peer-to-peer applications.
In Chapter \ref{ch:evaluation} we show how we test correctness and
performance of Ghoul.
In Chapter \ref{ch:conclusion} we summarize our contribution and state future
plans towards the prototype we have developed.

\section{Contributions}
This thesis has the following contributions:
\begin{itemize}
  \item Design of Ghoul --- a security extension of the Kademlia protocol that
    protects against malicious attacks.
  \item Ghoul's prototype --- Java implementation of the Kademlia protocol with
    message extensions, certificate management, and registration mechanisms.
    Ghoul is an open-source project.
    An up-to-date Ghoul's code is available for public audience at \url{https://github.com/gregorias/ghoul}.
  \item Design and implementation of dfuntest --- a framework for automation of
    distributed tests of Peer-to-Peer applications.
    Dfuntest is available as a separate project at
    \url{https://github.com/gregorias/dfuntest}.
\end{itemize}

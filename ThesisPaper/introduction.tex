\chapter{Introduction}

Peer-to-Peer applications are applications which run on multiple computers in
such a way that each instance is treated equally.
This is in direct opposition to a client-server model, such as for example
web-browsing, in which one instance (the client) request data from another (the
server), which handles and serves the data.
Peer-to-Peer model is popular among applications that need to scale dynamically
with the size of their user-base and applications for which users want privacy
and independence of central authority.
A good example of both of those factors is BitTorrent.
BitTorrent is a file-sharing protocol.
BitTorrent application wanting to download a file, acquires a list of other
computers that are interested in that file (either want to download it or have
it).
Next, the application downloads missing parts of the file from hosts that have
it, while at the same time providing parts of the file it has to other hosts.
This kind of sharing approach means that with the increase of an interest in a
file the total bandwidth of hosts having the file also increases.
Since there is no central-authority, noone may easily collect data about user
behaviour or shutdown the service.

An important part of a Peer-to-Peer application is how different instances
communicate with each other.
Unlike the client-server model, in which there is global known, always-present
authority to communicate to, Peer-to-Peer applications have to handle users
dynamically joining and leaving the network (this is called a churn) and find
a way to efficiently send messages between two arbitrary nodes.
This problem may be compared the problem of sending a message to an unknown
town.
We are in a town X and want to send a message to town Z.
However we only how to reach our neighbouring town and we do not know the entire
map of the country.
One of the roles of a Peer-to-Peer application is to define a protocol for
creation of roads in the country (the entire map is called a topology) and
establishment of routes to between two non-neighbouring towns.
A popular approach is to use structured topologies created by Distributed Hash
Tables (DHT). 
Distributed Hash Tables define where roads are created and how two entities may
find each other.
DHTs additionally provide constrainst on which nodes may be directly connected.
The structure that they create has certain properties that guarantee that routes
between non-neighbouring nodes will be relatively short and easy to find.

Unfortunately the structural nature of the topology created by the DHT and their
Peer-to-Peer character make them vulnerable to attacks.
[TODO examples]
Since topology is the basis of a Peer-to-Peer application, a successfull attack
on it is often enough to shut it down entirely.

In this work we show 



\section{Overview}
In chapter \ref{ch:threats} the problem domain.
We define a Distributed Hash Table (DHT) and evaluate its typical weaknesses.
We show what are the main classes of security threats that DHTs are vulnerable to.
Finally we review proposed security mechanisms.

In chapter \ref{ch:description} we describe the protocols used in Ghouls.

In chapter \ref{ch:dfuntest} we present dfuntest - a framework for testing distributed peer-to-peer applications.
Dfuntest was originally developed by the author to test kademlia application in Planet-lab environment, but its design later proved to be flexible enough to be a general framework for any P2P application.

In chapter \ref{ch:implementation} we explain how Ghoul is built and run as a stand-alone application.
We focus on presenting tools and techniques used to guarantee high-quality code.

In chapter \ref{ch:conclusion} we summarize our contribution and state future plans towards the prototype we have developed.

\section{Contributions}
This paper provides the following contributions:
\begin{itemize}
  \item Design of Ghoul - 
\end{itemize}
